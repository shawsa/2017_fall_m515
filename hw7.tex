\documentclass[12pt]{article}

%***************************************************************************************************
% Math
\usepackage{fancyhdr} 
\usepackage{amsfonts}
\usepackage{amsmath}
\usepackage{amssymb}
\usepackage{amsthm}
\usepackage{dsfont}

%***************************************************************************************************
% Macros
\usepackage{calc}

%***************************************************************************************************
% Commands and Custom Variables	
\newcommand{\problem}[1]{\hspace{-4 ex} \large \textbf{Problem #1} }
\let\oldemptyset\emptyset
\let\emptyset\varnothing
\newcommand{\norm}[1]{\left\lVert#1\right\rVert}
\newcommand{\sint}{\text{s}\kern-5pt\int}
\newcommand{\powerset}{\mathcal{P}}
\renewenvironment{proof}{\hspace{-4 ex} \emph{Proof}:}{\qed}
\newcommand{\RR}{\mathbb{R}}
\newcommand{\NN}{\mathbb{N}}
\newcommand{\QQ}{\mathbb{Q}}
\newcommand{\ZZ}{\mathbb{Z}}
\newcommand{\CC}{\mathbb{C}}


%***************************************************************************************************
%page
\usepackage[margin=1in]{geometry}
\usepackage{setspace}
%\doublespacing
\allowdisplaybreaks
\pagestyle{fancy}
\fancyhf{}
\rhead{Shaw \space \thepage}
\setlength\parindent{0pt}

%***************************************************************************************************
%Code
\usepackage{listings}
\usepackage{courier}
\lstset{
	language=Python,
	showstringspaces=false,
	formfeed=newpage,
	tabsize=4,
	commentstyle=\itshape,
	basicstyle=\ttfamily,
}

%***************************************************************************************************
%Images
\usepackage{graphicx}
\graphicspath{ {images/} }
\usepackage{float}

%tikz
\usepackage[utf8]{inputenc}
\usepackage{pgfplots}
\usepgfplotslibrary{groupplots}

%***************************************************************************************************
%Hyperlinks
%\usepackage{hyperref}
%\hypersetup{
%	colorlinks=true,
%	linkcolor=blue,
%	filecolor=magenta,      
%	urlcolor=cyan,
%}

\begin{document}
	\thispagestyle{empty}
	
	\begin{flushright}
		Sage Shaw \\
		m515 - Fall 2017 \\
		\today
	\end{flushright}
	
{\Large \textbf{HW 7}}\bigbreak

%***************************************************************************************************
\problem{13.1 (Tao, ex 1.44, 1.45)}
Let $A_n$ be measurable sets and assume that $\sum m(A_n)<\infty$. Show that almost every $x\in\RR^d$ is contained in at most finitely many of the $A_n$. [Hint: Use Tonelli's theorem on the functions $\chi_{A_n}$.] This is the Borel--Cantelli lemma.
Give a counterexample to the above conclusion, showing that the hypothesis $\sum m(A_n)<\infty$ cannot be replaced by the weaker condition $\lim m(A_n)=0$. \bigbreak

	\begin{proof}
		Note that $m(\chi_{A_n}) = \int \chi_{A_n}$. By Tonelli's theorem we know that $\int \sum \chi_{A_n} = \sum \int \chi_{A_n} = \sum m(A_n) < \infty$. By exercise 11.2 from the last homework, $\sum \chi_{A_n} < \infty$ almost everywhere. For every $x \in \RR^d$, $\sum \chi_{A_n}(x) =$ the count of how many $A_n$ that contain $x$. This shows that almost every $x$ is contained in at most finitely many $A_n$. \bigbreak
	\end{proof}

	\bigbreak
	
	For the counter example consider the following sequence of sets. For every $ n \in \NN$ let $m$ be the largest integer such that $2^m \leq n$ and let $r = n - 2^m$ so that $2^m + r = n$. Then define $A_n = [r\tfrac{1}{2^m}, (1+r)\tfrac{1}{2^m}]$. A few examples follow
	\begin{align*}
		A_1 & = [0,1] \\
		A_2 & = [0, \tfrac{1}{2}] & A_3 & = [\tfrac{1}{2}, 1]\\
		A_4 & = [0, \tfrac{1}{4}] & A_5 & = [\tfrac{1}{4}, \tfrac{2}{4}] & A_6 & = [\tfrac{2}{4}, \tfrac{3}{4}] & A_7 & = [\tfrac{3}{4}, \tfrac{4}{4}] \\
	\end{align*}
	Clearly $m(A_n) = \tfrac{1}{2^m}$ and $\lim m(A_n)=0$, but, each $x \in [0,1]$ is in infinitely many $A_n$.

\bigbreak


%***************************************************************************************************
\problem{Ex 14.1 (Tao, ex 1.4.26)} Let $\mu$ be a measure on $(X,\mathcal B)$. Show that $\mathcal B$ can be extended to $\sigma$-algebra $\hat{\mathcal B}$ and $\mu$ to a measure $\hat\mu$ on $(X,\hat{\mathcal B})$ in such a way that $\hat\mu$ is complete. \bigbreak

	\begin{proof}
		Let $\mathcal{N}$ be the $\sigma$-algebra generated by subsets of null-sets under $\mu$. Define $\hat{\mathcal{B}} = \{B \cup N| B \in \mathcal{B}, N \subseteq A \in \mathcal{B}$ where $\mu(A)=0 \}$. We now need to show that $\hat{\mathcal{B}}$ is a $\sigma$-algebra, by showing it's closed under countable unions and complements. Let $B_n\cup N_n \in \hat{\mathcal{B}}$ for $n \in \NN$. Then
		$$
		\bigcup B_n\cup N_n = \bigcup B_n \cup \bigcup N_n
		$$
		Since $\mathcal{B}$ and $N$ are sigma algebras, $\hat{\mathcal{B}}$ is closed under countable union. \bigbreak
		
		Let $B \cup M \in \hat{\mathcal{B}}$ where $B \in \mathcal{B}$ and $M \in N$. Without loss of generality let $B$ and $M$ be disjoint. Then There exists some $A \in \mathcal{B}$ such that $M \subseteq A$ and $\mu(A) = 0$. Then $B \cup M = (B \cup A) \setminus (A \setminus M)$. Then
		$$
		(B \cup M)^C = (B \cup A)^C \cup (A \setminus M)
		$$
		Since $(B \cup A) \in \mathcal{B}$ and $(A \setminus M) \subseteq A$, we have $(B \cup M)^C \in \hat{\mathcal{B}}$. \bigbreak
		
		Now define $\hat{\mu}: \hat{\mathcal{B}} \to [0,\infty]$ by $\hat{\mu}(A) = \{\inf \mu(B) | A \subseteq B, B \in \mathcal{B}\}$. This is well-defined because infima are well defined, and everything in $\hat{\mathcal{B}}$ is a subset of at least $X$. We would like to show that $\hat{\mu}$ extends $\mu$. Let $B \in \mathcal{B}$ and let $A \in \mathcal{B}$ such that $B \subseteq A$. Then $A \setminus B \in \mathcal{B}$ and $\mu(A) = \mu(A \setminus B) + \mu(B)$. Since $\mu$ is always non-negative $\mu(A) \geq \mu(B)$. Thus $\hat{\mu}(B) = \mu(B)$.
		
		Now we need to show that $\hat{\mu}$ is a measure. Clearly $\hat{\mu}(\emptyset) = 0$. Next suppose that $B_n \cup N_n$ are a countable sequence of disjoint sets in $\hat{\mathcal{B}}$. Without loss of generality let $B_n$ and $N_n$ be disjoint for each $n$ as well. By definition there exists some $A_n \in \mathcal{B}$ (wlog $A_n$ and $B_n$ disjoint) such that $\mu(A) = 0$ and $N_n \subseteq A_n$.
		$$
		\hat{\mu} \Big( \bigcup B_n \cup \bigcup N_n \Big) \leq \mu(\bigcup B_n \cup \bigcup A_n) = \sum \mu(B_n)
		$$
		Every super set of $\bigcup B_n \cup \bigcup N_n$ also contains $\bigcup B_n$ so by definition, $\hat{\mu} \Big( \bigcup B_n) \leq \hat{\mu} \Big( \bigcup B_n \cup \bigcup N_n \Big)$. Since 
		$$
		\sum \mu(B_n) = \hat{\mu} \Big( \bigcup B_n) \leq \hat{\mu} \Big( \bigcup B_n \cup \bigcup N_n \Big) \leq \sum \mu(B_n)
		$$
		we know that $hat{\mu} \Big( \bigcup B_n \cup \bigcup N_n \Big) = \sum \mu(B_n)$. Also, since every measurable subset of a null-set is null, $\sum \hat{\mu}(N_n) = 0$ and $\hat{\mu}(N_n)=0$ for each $n$, so
		$$
		\hat{\mu} \Big( \bigcup B_n \cup \bigcup N_n \Big) = \sum \hat{\mu}(B_n \cup N_n)
		$$
	\end{proof}


\bigbreak
%***************************************************************************************************
\problem{Ex 14.2 (Tao, ex 1.4.49)} Let $f$ be a nonnegative Lebesgue measurable function. Show that $\mu(A)=\int f\chi_A$ is a measure. \bigbreak

	\begin{proof}
		First note that $\int f \chi_\emptyset = \int 0 = 0$. Now let $A_n$ be a countable collection of disjoint measurable sets. Then note that $\chi_{\bigcup A_n} = \sum \chi_{A_n}$. Finally
		\begin{align*}
			\mu \Big ( \bigcup A_n \Big ) & = \int f \chi_{\bigcup A_n} \\
			& = \int f \sum \chi_{A_n} \\
			& = \int \sum f \chi_{A_n} \\
			& = \sum \int f \chi_{A_n} \text{\ \ \ (by Tonelli's)} \\
			& = \sum \mu(A_n)
		\end{align*}
		Thus $\mu$ is a measure.
	\end{proof}

\bigbreak
%***************************************************************************************************
\problem{Ex 14.3 (Tao, ex 1.7.6)} Give an example of a finitely additive measure that is not a premeasure. [Hint: work on the measurable space $(\NN,\mathcal P(\NN))$ and define $\mu_0$ separately for finite and infinite sets.] \bigbreak

	Define the finitely additive measure $\mu$ on $(\NN, \powerset(\NN))$ by
	\[
		\mu(A) = 
			\begin{cases}
				0 \text{ if $A$ is finite}\\
				\infty \text{ if $A$ is infinite}
			\end{cases}
	\]
	This is obviously finitely additive since the union of two finite sets is finite, but not a premeasure. Let $A_n = \{n\}$. Then $\bigcup A_n = \NN$, but 
	$$
	\mu \Big( \bigcup A_n \Big ) = \mu(\NN) = \infty \neq 0 = \sum 0 = \sum \mu(A_n)
	$$

\end{document}
