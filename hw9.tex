\documentclass[12pt]{article}

%***************************************************************************************************
% Math
\usepackage{fancyhdr} 
\usepackage{amsfonts}
\usepackage{amsmath}
\usepackage{amssymb}
\usepackage{amsthm}
%\usepackage{dsfont}

%***************************************************************************************************
% Macros
\usepackage{calc}

%***************************************************************************************************
% Commands and Custom Variables	
\newcommand{\problem}[1]{\hspace{-4 ex} \large \textbf{Problem #1} }
\let\oldemptyset\emptyset
\let\emptyset\varnothing
\newcommand{\norm}[1]{\left\lVert#1\right\rVert}
\newcommand{\sint}{\text{s}\kern-5pt\int}
\newcommand{\powerset}{\mathcal{P}}
\renewenvironment{proof}{\hspace{-4 ex} \emph{Proof}:}{\qed}
\newcommand{\RR}{\mathbb{R}}
\newcommand{\NN}{\mathbb{N}}
\newcommand{\QQ}{\mathbb{Q}}
\newcommand{\ZZ}{\mathbb{Z}}
\newcommand{\CC}{\mathbb{C}}
\newcommand{\BB}{\mathcal{B}}
\newcommand{\SA}{\mathcal{A}}



%***************************************************************************************************
%page
\usepackage[margin=1in]{geometry}
\usepackage{setspace}
\doublespacing
\allowdisplaybreaks
\pagestyle{fancy}
\fancyhf{}
\rhead{Shaw \space \thepage}
\setlength\parindent{0pt}

%***************************************************************************************************
%Code
\usepackage{listings}
\usepackage{courier}
\lstset{
	language=Python,
	showstringspaces=false,
	formfeed=newpage,
	tabsize=4,
	commentstyle=\itshape,
	basicstyle=\ttfamily,
}

%***************************************************************************************************
%Images
\usepackage{graphicx}
\graphicspath{ {images/} }
\usepackage{float}

%tikz
\usepackage[utf8]{inputenc}
\usepackage{pgfplots}
\usepgfplotslibrary{groupplots}

%***************************************************************************************************
%Hyperlinks
%\usepackage{hyperref}
%\hypersetup{
%	colorlinks=true,
%	linkcolor=blue,
%	filecolor=magenta,      
%	urlcolor=cyan,
%}

\begin{document}
	\thispagestyle{empty}
	
	\begin{flushright}
		Sage Shaw \\
		m515 - Fall 2017 \\
		\today
	\end{flushright}
	
{\large \textbf{HW 9}}\bigbreak

%%%%%%%%%%%%%%%%%%%%%%%%%%%%%%%%%%%%%%%%%%%%%%%%%%%%%%%%%%%%%%%%%%%%%%%%%%%%%%%%%%%%%%%%%%%%%%%%%%%%
\problem{17.2 (12:1.3)}
Ex 17.2 (12:1.3). Show that the unit ball of a normed vector space is convex. That is, for $x,y$ in the ball and $\lambda\in(0,1)$ we have $\lambda x+(1-\lambda)y$ is also in the ball.

	\begin{proof}
		Let $x,y$ be vectors in the normed vector space such that $\norm{x},\norm{y} \leq 1$, and let $\lambda \in (0,1)$. Without loss of generaltiy, suppose that $\norm{y} \leq \norm{x}$. Then $\lambda x + (1 -\lambda)y$ is clearly also a vector and
		\begin{align*}
			\norm{\lambda x + (1 -\lambda)y} & \leq \norm{\lambda x} + \norm{(1 -\lambda)y} \\
			& = \lambda\norm{ x} + (1 -\lambda)\norm{y} \\
			& \leq \lambda\norm{ x} + (1 -\lambda)\norm{x} \\
			& = \norm{x} \\
			& \leq 1
		\end{align*}
	\end{proof}


%%%%%%%%%%%%%%%%%%%%%%%%%%%%%%%%%%%%%%%%%%%%%%%%%%%%%%%%%%%%%%%%%%%%%%%%%%%%%%%%%%%%%%%%%%%%%%%%%%%%
\problem{17.3 (12:3.1)}
Ex 17.3 (12:3.1). Consider the operators $D(f)=f^\prime$, $(Sf)(x)=\int_a^x fd\mu$, and $I(f)=\int fd\mu$. What are appropriate domains and codomains of each operator? Show that $S$ and $I$ are continuous, and $D$ is not continuous. \bigbreak

	\textbf{$D$:}\\
	The domain of $D$ can at most be the set of differentiable functions. A small part of me thinks that we may need to restrict it to functions differentiable on some closed interval $[a,b]$, but I'm not sure. This operator is discontinuous. \\
	\begin{proof}
		Consider the functions $f_n(x) = \frac{1}{1+nx^2}$. For all of these functions $\norm{f_n} = 1$. Wolfram Alpha gives the derivatives of $f_n$ as 
		\begin{align*}
			f_n^\prime(x) = -2nx(1+nx^2)^{-2} \\
			f_n^{\prime\prime} = 2n(3nx^2-1)(nx^2+1)^{-3}
		\end{align*} 
		We can see that $f_n^{\prime\prime} = 0$ when $3nx^2-1 = 0$. Thus $f^\prime$ has relative extrema at the points $x = \pm \sqrt{\frac{1}{3n}}$. Inserting these points into $f^\prime$ we get that $f^\prime$ has extreme values of
		$$
		f^\prime \Big(\pm \sqrt{\tfrac{1}{3n}} \Big) = -2n \Big(\pm \sqrt{\tfrac{1}{3n}} \Big) \Big(1+n\tfrac{1}{3n} \Big)^{-2} = C\sqrt{n}
		$$
		for a constant $C$ that we need not compute. Since the norm of the deriviative is dependent on $n$ which can get arbitrarily large, $D$ is unbounded and hence not continuous.
	\end{proof}

%%%%%%%%%%%%%%%%%%%%%%%%%%%%%%%%%%%%%%%%%%%%%%%%%%%%%%%%%%%%%%%%%%%%%%%%%%%%%%%%%%%%%%%%%%%%%%%%%%%%
\problem{18.2 (BBT, ex 12:5.2)}
Ex 18.2 (BBT, ex 12:5.2). Let $\ell^\infty$ denote the space of bounded real sequences with the supremum norm, and let $c$ denote the subspace of convergent real sequences. Define $p$ on $\ell^\infty$ by
$p(x)=\limsup_{n\to\infty}\frac{x_1+\cdots+x_n}{n}$ .
Verify that $p$ is a sublinear functional such that $\lim x\leq p(x)$. If we apply the Hahn--Banach theorem to obtain a bounded linear functional $L$ extending $\lim$, show that $\liminf x\leq L(x)\leq\limsup x$ and calculate the value of $L(0,1,0,1,\ldots)$.

	\begin{proof}
		Given sequences $x,y \in \ell^\infty$, and a scalar $c \in \RR$, 
		\begin{align*}
			p(cx) & = \limsup_{n\to\infty}\frac{cx_1+\cdots+cx_n}{n} \\
			& = c \limsup_{n\to\infty}\frac{x_1+\cdots+x_n}{n} \\
			& = cp(x)
		\end{align*}
		and
		\begin{align*}
			p(x+y) & = \limsup_{n\to\infty}\frac{x_1+y_1+\cdots+x_n+y_n}{n} \\
			& = \limsup_{n\to\infty}\frac{x_1+\cdots+x_n}{n} + \frac{y_1+\cdots+y_n}{n}\\
			& \leq \limsup_{n\to\infty}\frac{x_1+\cdots+x_n}{n} + \limsup_{n\to\infty} \frac{y_1+\cdots+y_n}{n}\\
			& = p(x) + p(y)
		\end{align*}
		Thus $p$ is a sublinear functional. \\
		Suppose that $\lim x = L$. Let $\epsilon > 0$ be given. Then there exists an $N$ such that if $N \leq n$ then $\vert x_n - L \vert < \epsilon$. Then
		\begin{align*}
			\lim & \frac{x_1 + \dots +x_N + x_{N+1} + \dots + x_n}{n}\\
			& = \lim \frac{x_1 + \dots + x_n}{n} -L +L \\
			& = L + \lim \frac{x_1 + \dots + x_n}{n} -\frac{nL}{n} \\
			%& = L + \lim \frac{x_1 + \dots +x_N + x_{N+1} + \dots + x_n - nL}{n} \\
			& = L + \lim \frac{(x_1-L) + \dots  + (x_n-L)}{n} \\
			%& = L + \lim \frac{(x_1-L) + \dots + (x_N-L) + (x_{N+1}-L) + \dots + (x_n-L)}{n} \\
			& = L + \lim \frac{(x_1-L) + \dots + (x_N-L)}{n} + \lim \frac{(x_{N+1}-L) + \dots + (x_n-L)}{n} \\
			& = L + \lim \frac{(x_{N+1}-L) + \dots + (x_n-L)}{n}
		\end{align*}
		In considering the above limit, note that
		$$
		-\epsilon = \frac{ -\epsilon - \dots	- \epsilon}{n} < \frac{(x_{N+1}-L) + \dots + (x_n-L)}{n} < \frac{ \epsilon + \dots	+ \epsilon}{n} = \epsilon
		$$
		Taking $\epsilon$ to zero we have that $\lim x = p(x)$. Thus $\lim x \leq p(x)$. 
		
		\textbf{******************* the other stuff  ****************}
	\end{proof}

%%%%%%%%%%%%%%%%%%%%%%%%%%%%%%%%%%%%%%%%%%%%%%%%%%%%%%%%%%%%%%%%%%%%%%%%%%%%%%%%%%%%%%%%%%%%%%%%%%%%
\problem{18.3}
Ex 18.3. Let $\RR^d$ be equipped with any norm that makes it into a normed vector space. Show that every linear functional on $\RR^d$ is continuous.

	\begin{proof}
		We know that every linear functional on $\RR^d$ is a dual vector, so it is sufficient to show that any dual vector $T=[a_1\ a_2\ \dots a_d]$ is continuous. 
	\end{proof}
\end{document}