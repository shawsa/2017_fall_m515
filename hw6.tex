\documentclass[12pt]{article}


% Math		****************************************************************************************
\usepackage{fancyhdr} 
\usepackage{amsfonts}
\usepackage{amsmath}
\usepackage{amssymb}
\usepackage{amsthm}
%\usepackage{dsfont}

% Macros	****************************************************************************************
\usepackage{calc}

% Commands and Custom Variables	********************************************************************
\newcommand{\problem}[1]{\hspace{-4 ex} \large \textbf{Problem #1} }
\let\oldemptyset\emptyset
\let\emptyset\varnothing
\newcommand{\norm}[1]{\left\lVert#1\right\rVert}
\newcommand{\sint}{\text{s}\kern-5pt\int}
\newcommand{\powerset}{\mathcal{P}}
\renewenvironment{proof}{\hspace{-4 ex} \emph{Proof}:}{\qed}

%page		****************************************************************************************
\usepackage[margin=1in]{geometry}
\usepackage{setspace}
\doublespacing
\allowdisplaybreaks
\pagestyle{fancy}
\fancyhf{}
\rhead{Shaw \space \thepage}
\setlength\parindent{0pt}

%Code		****************************************************************************************
\usepackage{listings}
\usepackage{courier}
\lstset{
	language=Python,
	showstringspaces=false,
	formfeed=newpage,
	tabsize=4,
	commentstyle=\itshape,
	basicstyle=\ttfamily,
}

%Images		****************************************************************************************
\usepackage{graphicx}
\graphicspath{ {images/} }

%tikz
\usepackage[utf8]{inputenc}
\usepackage{pgfplots}

%***************************************************************************************************
%Hyperlinks	****************************************************************************************
%\usepackage{hyperref}
%\hypersetup{
%	colorlinks=true,
%	linkcolor=blue,
%	filecolor=magenta,      
%	urlcolor=cyan,
%}


\begin{document}
	\thispagestyle{empty}
	
	\begin{flushright}
		Sage Shaw \\
		m515 - Fall 2017 \\
		\today
	\end{flushright}
	
{\large \textbf{HW 6}}\bigbreak

%***************************************************************************************************
\problem{11.1} Let $f$ be a nonnegative measurable function on $\mathbb{R}$. Show that $\int f$ is equal to the $2$-dimensional Lebesgue measure of the region $R = \{(x,y)\mid 0\leq y \leq f(x) \}$
	
	\begin{proof}
		First suppose that $f$ is bounded with bounded support. Then $\int f = \bar{\int} f$. Consider a simple function $g$ such that $f \leq g$. Since $g$ is simple, it can be written $g = \sum\limits_{n=1}^k c_n \chi_{A_n}$. Where each $A_n$ is disjoint and Lebesgue measurable. Then each $A_n$ can be written as the countable union of almost disjoint intervals (boxes in $\mathbb{R}$) $A_n = \bigcup\limits_{i \in \mathbb{N}}I_{i,n}$. Consider the countable union of almost disjoint boxes $A = \bigcup\limits_{i \in \mathbb{N}} \bigcup\limits_{n=1}^k c_n \times I_{i,n}$. It's easily seen that $R \subseteq A$ and 
		$$
		\sum\limits_{i \in \mathbb{N}} \sum\limits_{n=1}^k \text{vol}\big( c_n \times I_{i,n} \big) = m(A) =  \sum\limits_{n=1}^k c_n \Bigg( \sum\limits_{i \in \mathbb{N}} m(I_{i,n}) \Bigg) = \sint g
		$$
		Then clearly $m(R) \leq \bar{\int}f = \int f$. Suppose the support is bounded by $[-M,M]$, and $f$ is bounded by $N$. Then for any union of boxes $A$ such that $A \subseteq R$, $B \subseteq R$ and $m(B) \leq m(R)$ where
		$$
		B =  A \cap \big([-M,M] \times [0,N] \big)
		$$
		This is simply the region below a simple function for which the measure is the simple integral. Thus we have $\bar{\int}f \leq m(R)$, and thus $\int f = m(R)$. \bigbreak
		
		 Applying the truncation lemmas we generalize to the case where $f$ is not necessarily bounded, with potentially unbounded support.
		
		
	\end{proof}

%***************************************************************************************************
\problem{11.2 (i)} Let $f$ be an nonnegative measurable function on $\mathbb{R}^d$. Show that if $\int f<\infty$ then $f$ is finite almost everywhere.

	\begin{proof}
 		Suppose that $\int f<\infty$ for some nonnegative measurable function $f$. Supose that $f$ were infinite on a set $A$ of positive measure. Then the simple function $\infty \chi_A \leq f$ but $\sint \infty \chi_A = \infty$ and thus $ \infty \leq \int f$ contradicts our premise. Thus $f$ is finite almost everywhere.
 	\end{proof}
 		
\problem{11.2 (ii)} Give a counterexample to show that the converse is false. \bigbreak

 		Consider the function $f(x) = \sum\limits_{n\in \mathbb{N}} \chi_{[n, \infty)}$ (infitie steps). This function is finite everywhere but the integral is infinite. \bigbreak
 		
\problem{11.2 (iii)} Show that $\int f=0$ if and only if $f=0$ almost everywhere.
	\begin{proof}
 		Suppose that $f(x)=0$ almost everywhere. Let $g \equiv 0$. Note that since $g = 0$ almost everywhere (everywhere in fact) that $\int f = \int g$ by proposition 11.2. Thus $\int f = 0$.
 		
 		Suppose now that $\int f = 0$. Suppose also that it is not the case that $f=0$ almost everywhere. In other words, let $A = \{x \vert f(x)>0\}$ and suppose that $m(A) > 0$. Let $A_n = f^{-1} \Big( (\tfrac{1}{n},\infty) \Big)$ for all $n \in \mathbb{N}$. Since $f$ is measurable and $(\tfrac{1}{n},\infty)$ is open, each $A_n$ is measurable. Also, $A_n \subseteq A_{n+1}$. Then by upwards monotone convergence
 		$$
 		0< m(A) = m \Big( \bigcup_{n \in \mathbb{N}} A_n \Big ) = \lim_{n \to \infty} m(A_n)
 		$$
 		If $m(A_n)=0$ for all $n \in \mathbb{N}$ then we would have a contradiction, therefore there is some $N \in \mathbb{N}$ such that $m(A_n)>0$. Then the simple function $g = \tfrac{1}{N}\chi_{A_N} \leq f$ and $\sint g = \tfrac{1}{N} m(A_N) > 0$ and then $\int f >0$. Finally, by contradiction we have that $f=0$ almost everywhere.
 		
	\end{proof}

%***************************************************************************************************
\problem{11.3} Give an example of a nonnegative function which is measurable, but has different lower and upper Lebesgue integrals.

	Consider the function 
	\[
	f(x) = \begin{cases}
		x^{-\frac{1}{2}} \text{ if } 0<x\\
		0 \text{ else}
	\end{cases}
	\]
	It can be shown that $ \underline{\int}f = \int f = 2$. Since $f$ is unbounded in any neighborhood about $0$ any simple function that is greater than or equal to $f$ will have a region of positive measure over which it assumes the value $\infty$. This will result in $\bar{\int}f = \infty$.
	

%***************************************************************************************************
\problem{12.1} Let $f$ be absolutely integrable. Show that for any $\epsilon>0$ there exists a bounded measurable set $A$ such that $\int |f|\chi_{A^c}<\epsilon$.

	\begin{proof}
		Let $f$ be absolutely integrable. Let $\epsilon > 0 $ be given. Define $A_n = \{ \vec{x} \in \mathbb{R}^d \vert \norm{\vec{x}} < n \}$ (open ball of radius $n$) for every $n \in \mathbb{N}$. Clearly each $A_n$ is bounded and measurable. Define $f_n = \vert f \vert \chi_{A_n}$. Then $f_n$ converges to $\vert f \vert$ and $f_n \leq f_{n+1}$ so by monotone convergence we can say 
		$$
		\int \vert f \vert = \int \lim_{n \to \infty}{f_n} = \lim_{n \to \infty} \int f_n
		$$
		Then there exists some $N \in \mathbb{N}$ such that for all $n \geq N, \int \vert f \vert - \int f_n < \epsilon$. Let $A = A_N$. Then 
		$$
		\int |f|\chi_{A^c} = \int \vert f \vert - |f|\chi_{A} = \int \vert f \vert - \int f_N < \epsilon 
		$$
	\end{proof}

%***************************************************************************************************
\problem{12.2} Let $f$ be a finite measurable function with finite measure support. Show that for any $\epsilon>0$, there exists a measurable set $A$ such that $m(A)<\epsilon$ and $f$ is locally bounded outside of $A$. In other words, for every $n$ there exists $M$ such that for all $x\in[-n,n]^d\setminus A$ we have $ \vert f(x) \vert \leq M$.\bigbreak

	\begin{proof}
		Let $f$ be a finite measurable function. Let $S$ be the support of $f$ and let $m(S) = K$ for some finite $K$. Let $\epsilon > 0$ be given.\bigbreak
		First suppose that $f$ is a real-valued, non-negative function with a bounded support.  Define $B_n = f^{-1}\Big( (0,n) \Big)$. Clearly $B_n \subseteq B_{n+1}$ and $\lim\limits_{n \to \infty} B_n = S$. By the monotone convergence theorem we have that $\lim\limits_{n \to \infty} m(B_n) = m(S)$. Then there exists some $N_1$ such that for all $n \geq N_1, m(S) - m(B_n) < \epsilon$. Define $A = S \setminus B_{N_1}$. Since $B_n \subseteq S$ we know that $m(A) < \epsilon$. Since we supposed that the support of $f$ was bounded, let $S \subseteq [-M,M]$. Then for all $x\in[-M,M]^d\setminus A$ we have $ \vert f(x) \vert \leq N_1$.\bigbreak
		
		Now consider the case when $S$ is not bounded. Let $D_n = [-n,n]^d \cap S$. Clearly $D_n \subseteq D_{n+1}$ and $\cup D_n = S$. Then by monotone convergence we know that $\lim\limits_{n \to \infty} m(D_n) = m(S)$. Then there exists some $N_2$ such that $m(S \ D_{N_2}) < \tfrac{\epsilon}{2}$. Consider $f$ restricted to the domain of $D_{N_2}$. Then there is some $A_1 \subseteq D_{N_2}$ such that $m(A_1) < \tfrac{\epsilon}{2}$ and if $x \in D_{N_1}$ then $ \vert f(x) \vert \leq N_1$ for some $N_1$. Define $A = A_1 \cup (S \ D_{N_2})$. Then $m(A) < \epsilon$ and $f$ is locally bounded outside of $A$. \bigbreak
		
		To extend this argument to the general case, we first apply it to the case when $f$ is real valued by decomposing it like so $f = f^+ - f^-$ and then to the general case by noting that $f = \text{Re}(f) + i \text{Im}(f)$. In each generalization the argument is essentially the same.
	\end{proof}

\end{document}
