\documentclass[12pt]{article}

\usepackage{fancyhdr} 
\usepackage{amsfonts}
\usepackage{amsmath}
\usepackage{amsthm}
\usepackage{dsfont}

\pagestyle{fancy}
\fancyhf{}
\rhead{Shaw \space \thepage}

\setlength\parindent{0pt}

\begin{document}
	\thispagestyle{empty}
	
\begin{flushright}
	Sage Shaw \\
	m515 - Fall 2017 \\
	Sep. 1, 2017
\end{flushright}
	
{\large \textbf{HW 2: 3.2,4,5; 4.2,3}}\bigbreak

%problem 3.2
\hspace{-4 ex}\textbf{3.2} Show that $E$ is Jordan measurable iff for all $\epsilon>0$ there is an elementary set $A$ such that $m(E\bigtriangleup A)<\epsilon$. \bigbreak

	\begin{proof}\text{ }\\
		Let $E$ be a Jordon Measurable set, and let $\epsilon >0$ be given. Then by Lemma 3.2, part (1), we can say $\exists A,B$ such that $A \subseteq E \subseteq B$ where $A$ and $B$ are elementary sets, and $m(B\setminus A)< \epsilon$. \\
		Since $E \subseteq B$, $(E \setminus A) \subseteq (B \setminus A)$. \\
		Since $A \subseteq E$, $A \setminus E = \emptyset$, and $E \bigtriangleup A = E \setminus A$. \\ 
		So $(E \bigtriangleup A) = (E \setminus A) \subset (B \setminus A)$.\\
		Then by Proposition 2.5, $m(E \bigtriangleup A) \leq m(B \setminus A) < \epsilon$.\bigbreak
		
		Now we prove the other direction. Suppose that for all $\epsilon>0$ there is an elementary set $A$ such that $m(E\bigtriangleup A)<\epsilon$.\\
		
		Let $\epsilon > 0$ be given. \\
		By our premise, $\exists $ an elementary set $A$ such that $m(E\bigtriangleup A)<\epsilon / 2$. \\
		Since $m(E \bigtriangleup A)$ exists, there exist elementary sets $C$ and $D$ such that \\
		$m(D \setminus C) < \epsilon/2$ and $C \subseteq E \bigtriangleup A \subseteq D$. \\
		Note that since $C \subseteq E \bigtriangleup A$, by monotonicity we can say that $m(C) < \epsilon/2$. \\
		Since $C \subseteq D$, we can say that $m(D)-m(C) = m(D \setminus C) < \epsilon/2$. \\
		Then $m(D) < \epsilon/2 + m(C) < \epsilon$. \bigbreak
		
		Consider the set $A \setminus D$. \\
		Since $E \bigtriangleup A \subseteq D$, $E \setminus A \subseteq D$ and thus we know that $A \setminus D \subseteq A \setminus (E \setminus A) \subseteq E$. Hence $A \setminus D$ is an elementary set contained in $E$. \\
		
		Now consider the elementary set $A \cup D$. \\
		Since $E \bigtriangleup A \subseteq D$, we can say $(E \setminus A) \subseteq D$ and $E \subseteq A \cup (E \setminus A) \subseteq A \cup D$. \\
		Hence $A \cup D$ is an elementary set that contains $E$. \\
		
		Note that $(A \cup D) \setminus (A \setminus D) = D$.\\
		So $m((A \cup D) \setminus (A \setminus D)) = m(D) < \epsilon$, where $(A \setminus D) \subseteq E \subseteq (A \cup D)$, and $(A \setminus D)$ and $(A \cup D)$ are elementary sets. By part 1 of lemma 3.2, $E$ is Jordan measurable. \\		
	\end{proof}

%problem 3.4
\hspace{-4 ex}\textbf{3.4} Say that $E$ is \emph{Jordan null} if $E$ is Jordan measurable and $m(E)=0$. Show that any subset of a Jordan null set is a Jordan null set. \bigbreak

	\begin{proof}\text{ }\\
		Let $A \subseteq E$ where $E$ is Jordan Measurable and $m(E)=0$. \\
		Let $\epsilon >0$ be given.\\
		Since $\emptyset \subseteq A$, and $m(\emptyset)=0$,  $0 \leq m_{*j}(A)$.\\
		Since $E$ is Jordon Measurable, $\exists$ elementary sets $A^{\prime}, B$ where $A^{\prime} \subseteq E \subseteq B$ such that $m(B \setminus A^{\prime})<\epsilon$. \\
		Then $m(B \setminus A^{\prime}) = m(B) - m(A^{\prime})$. \\
		Since $A^{\prime} \subseteq E$ and $A^\prime$ is elementary, $m(A^{\prime}) = 0$. \\
		Thus $m(B \setminus A^{\prime}) = m(B) < \epsilon$. \\
		Since $A \subseteq B, m^{*j}(A) \leq m(B) < \epsilon$. \\
		Taking the limit we can see that $0 \leq m_{*j}(A) \leq m^{*j}(A) \leq 0$, and $m(A)=0$.\\
		Thus $A$ is a Jordan null set.\\	
	\end{proof}



%problem 3.5
\hspace{-4 ex}\textbf{3.5} Show that the outer Jordan measure $m^{*j}(E)$ is equal to: $$x = \text{inf}\{\text{vol}(B_1)+\cdots+\text{vol}(B_k) : B_1,\ldots,B_k  \text{ are boxes} \}$$ and $E\subset \{B_1\cup\cdots\cup B_k\}$. \bigbreak

	\begin{proof}  \text{ }\\
		Define $S_1 =  \{\text{vol}(B_1)+\cdots+\text{vol}(B_k) : B_1,\ldots,B_k  \text{ are boxes} \}$ and $E\subset \{B_1\cup\cdots\cup B_k\}$. \\
		Then $x = \text{inf}(S_1)$. \\
		Also define $S_2 = \{m(A) : A \text{ is elementary and } A \subseteq E \}$. \\
		Then $m^{*j}(E) = \text{inf}(S_2)$ by definition. Also from using Lemma 2.2 and finite additivity it's clear that $S_2 \subseteq S_1$ (the measure of an elementary set is equal to the finite sum of measures of some boxes). Then $\text{inf}(S_1) \leq \text{inf}(S_2)$.\\
		
		Suppose that $x \in S_1$ and $ x < \text{inf}(S_2)$. Then $x = \sum\limits_{i=1}^k \text{vol}(B_i)$ for some $B_i$ boxes where $E \subseteq \bigcup\limits_{i=1}^{k}B_i$. Since elementary sets are closed under finite union,  $\bigcup\limits_{i=1}^{k}B_i$ is elementary, and by Lemma 2.2 it can be written as the finite union of \textit{disjoint} boxes $C_i$. So $x = \sum\limits_{i=1}^k \text{vol}(B_i) \geq m(\bigcup\limits_{i=1}^{k}B_i) = m(\bigcup\limits_{i=1}^{k}C_i) = \sum\limits_{i=1}^{k}m(C_i) \in S_1$. Then $x \geq \text{inf}(S_1)$ and we have a contradiction. \\
		Thus $\text{inf}(S_1) = \text{inf}(S_2) = m^{*j}(E)$. \\

		
%		Define $S_1^\prime =  \{\text{vol}(B_1)+\cdots+\text{vol}(B_k) : B_1,\ldots,B_k  \text{ are boxes} \}$ and $E\subset \{B_1\cup\cdots\cup B_k\}$. \\
%		Define $S_1 =  \{\text{vol}(B_1)+\cdots+\text{vol}(B_k) : B_1,\ldots,B_k  \text{ are boxes} \}$ and $E\subset \{B_1\cup\cdots\cup B_k\}$ where all $B_i$ are pairwise disjoint. \\
%		Clearly, $S_1 \subseteq S_1^\prime$ so $\text{inf}(S_1^\prime) \leq \text{inf}(S_1) $. \\
%		
%		Suppose that $x \in S_1^\prime$ and $ x < \text{inf}(S_1)$. Then $x = \sum\limits_{i=1}^k \text{vol}(B_i)$ for some $B_i$ boxes where $E \subseteq \bigcup\limits_{i=1}^{k}B_i$. Since elementary sets are closed under finite union,  $\bigcup\limits_{i=1}^{k}B_i$ is elementary, and by Lemma 2.2 it can be written as the finite union of \textit{disjoint} boxes $C_i$. So $x = \sum\limits_{i=1}^k \text{vol}(B_i) \geq m(\bigcup\limits_{i=1}^{k}B_i) = m(\bigcup\limits_{i=1}^{k}C_i) = \sum\limits_{i=1}^{k}m(C_i) \in S_1$. Then $x \geq \text{inf}(S_1)$ and we have a contradiction. Thus $\text{inf}(S_1^\prime) = \text{inf}(S_1)$.
%		
%		Also define $S_2 = \{m(A) : A \text{ is elementary and } A \subseteq E \}$. \\
%		By definition $\text{inf}(S_1) = x$ and $\text{inf}(S_2) = m^{*j}(E)$. \bigbreak
%		
%		It is sufficient to show that $S_1=S_2$. \bigbreak
%		
%		Let $x \in S_1$. Then there exist boxes $B_1,...,B_k$ such that $E\subset \bigcup\limits_{i=1}^k B_i$ and $x = \sum\limits_{i=1}^k \text{vol}(B_i)$. \\
%		By definition 
	\end{proof}



%problem 4.2
\hspace{-4 ex}\textbf{4.2} Let $f\colon[a,b]\to\mathbb{R}$. Show that if $f$ is continuous, then $f$ is Riemann integrable. Show that if $f$ is bounded and piecewise continuous, then $f$ is Riemann integrable. \bigbreak

	\begin{proof}  \text{ }\\
		Let $f : [a,b] \to \mathbb{R}$ be a continuous function. From Analysis we know that $f$ is also uniformly continuous and bounded.\\
		Let $\epsilon > 0$ be given. \\
		Since $f$ is uniformly continuous, there exists $ \delta >0$ such that for all $x,x^* \in [a,b]$, if $\vert x - x^* \vert < \delta$ then $\vert f(x) - f(x^*) \vert < \frac{\epsilon}{b-a}$.\\
		By the Archimedean property there exists some natural number $n$, such that $n > \frac{b-a}{\delta}$. \\
		Define $x_i = \frac{b-a}{n} i+ a$ for $i \in \{0,...,n\}$, and define $I_i = [x_{i-1}, x_i]$ for $i \in \{1,...,n\}$. Note that the length of each of these intervals is $\frac{b-a}{n} < \delta$ and there are $n$ intervals. \\
		Now define $\overline{h} = \sum\limits_i c_i \chi_{I_i}$ where $c_i = \text{max} \{f(x) : x \in I_i\}$. \\
		Also define $underline{h} = \sum\limits_i d_i \chi_{I_i}$ where $d_i = \text{min} \{f(x) : x \in I_i\}$. \\
		Since $f$ is bounded and continuous on $I_i$ we know that it has a maximum value and a minimum value over each interval and these is well defined. \\
		Clearly $\underline{h} \leq f \leq \overline{h}$ and both $\underline{h}$ and $\overline{h}$ are p.c. functions.\\
		
		Next we will show that $\underline{\int}f = \overline{\int}f$ using an $\epsilon$ style argument. \\
		
		From the definition of p.c. integral it is easy to see that $\text{pc}\int\underline{h} \leq \underline{\int}f \leq \underline{\int}f \leq \text{pc}\int\underline{h}$. \\
		
		$\underset{a}{b} \overbrace{a} \bar{A} \underline{\int} \overline{\int}$\\
		
		
		
	\end{proof}



%problem 4.3
\hspace{-4 ex}\textbf{4.3} Show that the Riemann integral satisfies the linearity and monotonicity properties. \bigbreak





\end{document}
