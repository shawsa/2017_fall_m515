\documentclass[12pt]{article}

\usepackage{fancyhdr} 
\usepackage{amsfonts}
\usepackage{amsmath}
\usepackage{amsthm}
\usepackage{dsfont}

\usepackage[margin=1in]{geometry}

\pagestyle{fancy}
\fancyhf{}
\rhead{Shaw \space \thepage}

\setlength\parindent{0pt}

\begin{document}
	\thispagestyle{empty}
	
\begin{flushright}
	Sage Shaw \\
	m515 - Fall 2017 \\
	Sep. 1, 2017
\end{flushright}
	
{\large \textbf{HW 2: 3.2, 4, 5; 4.2, 3}}\bigbreak

%problem 3.2
\hspace{-4 ex}\textbf{3.2} Show that $E$ is Jordan measurable iff for all $\epsilon>0$ there is an elementary set $A$ such that $m^{*J}(E\bigtriangleup A)<\epsilon$. \bigbreak

	\begin{proof}\text{ }\\
		Let $E$ be a Jordon Measurable set, and let $\epsilon >0$ be given. Then by Lemma 3.2, part (1), we can say $\exists A,B$ such that $A \subseteq E \subseteq B$ where $A$ and $B$ are elementary sets, and $m(B\setminus A)< \epsilon$. \\
		Since $E \subseteq B$, $(E \setminus A) \subseteq (B \setminus A)$. \\
		Since $A \subseteq E$, $A \setminus E = \emptyset$, and $E \bigtriangleup A = E \setminus A$. \\ 
		So $(E \bigtriangleup A) = (E \setminus A) \subseteq (B \setminus A)$.\\
		Then by Proposition 2.5, $m^{*J}(E \bigtriangleup A) = m(E \bigtriangleup A) \leq m(B \setminus A) < \epsilon$.\bigbreak
		
		Now we prove the other direction. Suppose that for all $\epsilon>0$ there is an elementary set $A$ such that $m(E\bigtriangleup A)<\epsilon$.\\
		
		Let $\epsilon > 0$ be given. \\
		
		By our premise, $\exists $ an elementary set $A$ such that $m^{*J}(E\bigtriangleup A)<\epsilon$. \\
		
		Then there exists some elementary set $D$ that such that $E \bigtriangleup A \subseteq D$ and $m^{*J}(E\bigtriangleup A) \leq m(D) < \epsilon$, since if there did not, then $\epsilon$ would be a lower bound greater than $m^{*J}(E\bigtriangleup A)$. \\
		
		Consider the set $A \setminus D$. \\
		Since $E \bigtriangleup A \subseteq D$, $A \setminus E \subseteq D$ and thus we know that $$A \setminus D \subseteq A \setminus (A \setminus E) = A \cap E \subseteq E$$
		Hence $A \setminus D$ is an elementary set contained in $E$. \\
		
		Now consider the elementary set $A \cup D$. \\
		Since $E \bigtriangleup A \subseteq D$, we can say $(E \setminus A) \subseteq D$ and 
		$$E \subseteq E \cup A = A \cup (E \setminus A) \subseteq A \cup D$$
		Hence $A \cup D$ is an elementary set that contains $E$. \\
		
		Note that $(A \cup D) \setminus (A \setminus D) = D$.\\
		So $m((A \cup D) \setminus (A \setminus D)) = m(D) < \epsilon$, where $(A \setminus D) \subseteq E \subseteq (A \cup D)$, and $(A \setminus D)$ and $(A \cup D)$ are elementary sets. By part 1 of lemma 3.2, $E$ is Jordan measurable. \\		
	\end{proof}

%problem 3.4
\hspace{-4 ex}\textbf{3.4} Say that $E$ is \emph{Jordan null} if $E$ is Jordan measurable and $m(E)=0$. Show that any subset of a Jordan null set is a Jordan null set. \bigbreak

	\begin{proof}\text{ }\\
		Let $A \subseteq E$ where $E$ is Jordan Measurable and $m(E)=0$. \\
		Let $\epsilon >0$ be given.\\
		Since $\emptyset \subseteq A$, and $m(\emptyset)=0$,  $0 \leq m_{*j}(A)$.\\
		Since $E$ is Jordon Measurable, $\exists$ elementary sets $A^{\prime}, B$ where $A^{\prime} \subseteq E \subseteq B$ such that $m(B \setminus A^{\prime})<\epsilon$. \\
		Then $m(B \setminus A^{\prime}) = m(B) - m(A^{\prime})$. \\
		Since $A^{\prime} \subseteq E$ and $A^\prime$ is elementary, $m(A^{\prime}) = 0$. \\
		Thus $m(B \setminus A^{\prime}) = m(B) < \epsilon$. \\
		Since $A \subseteq B, m^{*j}(A) \leq m(B) < \epsilon$. \\
		Taking the limit we can see that $0 \leq m_{*j}(A) \leq m^{*j}(A) \leq 0$, and $m(A)=0$.\\
		Thus $A$ is a Jordan null set.\\	
	\end{proof}



%problem 3.5
\hspace{-4 ex}\textbf{3.5} Show that the outer Jordan measure $m^{*j}(E)$ is equal to: $$x = \text{inf}\{\text{vol}(B_1)+\cdots+\text{vol}(B_k) : B_1,\ldots,B_k  \text{ are boxes} \}$$ and $E\subset \{B_1\cup\cdots\cup B_k\}$. \bigbreak

	\begin{proof}  \text{ }\\
		Define $S_1 =  \{\text{vol}(B_1)+\cdots+\text{vol}(B_k) : B_1,\ldots,B_k  \text{ are boxes} \}$ and $E\subset \{B_1\cup\cdots\cup B_k\}$. \\
		Then $x = \text{inf}(S_1)$. \\
		Also define $S_2 = \{m(A) : A \text{ is elementary and } A \subseteq E \}$. \\
		Then $m^{*j}(E) = \text{inf}(S_2)$ by definition. Also from using Lemma 2.2 and finite additivity it's clear that $S_2 \subseteq S_1$ (the measure of an elementary set is equal to the finite sum of measures of some disjoint boxes). Then $\text{inf}(S_1) \leq \text{inf}(S_2)$.\\
		
		Suppose that $x \in S_1$ and $ x < \text{inf}(S_2)$. Then $x = \sum\limits_{i=1}^k \text{vol}(B_i)$ for some $B_i$ boxes where $E \subseteq \bigcup\limits_{i=1}^{k}B_i$. Since elementary sets are closed under finite union,  $\bigcup\limits_{i=1}^{k}B_i$ is elementary, and by Lemma 2.2 it can be written as the finite union of \textit{disjoint} boxes $C_i$. So $x = \sum\limits_{i=1}^k \text{vol}(B_i) \geq m(\bigcup\limits_{i=1}^{k}B_i) = m(\bigcup\limits_{i=1}^{k}C_i) = \sum\limits_{i=1}^{k}m(C_i) \in S_1$. Then $x \geq \text{inf}(S_1)$ and we have a contradiction. \\
		Thus $\text{inf}(S_1) = \text{inf}(S_2) = m^{*j}(E)$. \\ 
	\end{proof}



%problem 4.2
\hspace{-4 ex}\textbf{4.2} Let $f\colon[a,b]\to\mathbb{R}$. Show that if $f$ is continuous, then $f$ is Riemann integrable. Show that if $f$ is bounded and piecewise continuous, then $f$ is Riemann integrable. \bigbreak

	\begin{proof}  \text{ }\\
		We first prove that if $f$ is continuous on $[a,b]$, then it is Riemann integrable over $[a,b]$. \\
		
		Let $f : [a,b] \to \mathbb{R}$ be a continuous function. From Analysis we know that $f$ is also uniformly continuous and bounded.\\
		Let $\epsilon > 0$ be given. \\
		Since $f$ is uniformly continuous, there exists $ \delta >0$ such that for all $x,x^* \in [a,b]$, if $\vert x - x^* \vert < \delta$ then $\vert f(x) - f(x^*) \vert < \frac{\epsilon}{b-a}$.\\
		By the Archimedean property there exists some natural number $n$, such that $n > \frac{b-a}{\delta}$. Define $x_i = \frac{b-a}{n} i+ a$ for $i \in \{0,...,n\}$, and define $I_i = [x_{i-1}, x_i]$ for $i \in \{1,...,n\}$. Note that these intervals cover $[a,b]$, the length of each of these intervals is $\frac{b-a}{n} < \delta$ and there are $n$ intervals. \\
		Now define $\overline{h} = \sum\limits_{i=1}^n c_i \chi_{I_i}$ where $c_i = \text{sup} \{f(x) : x \in I_i\}$. \\
		Also define $\underline{h} = \sum\limits_{i=1}^n d_i \chi_{I_i}$ where $d_i = \text{inf} \{f(x) : x \in I_i\}$. \\
		%Since $f$ is bounded and continuous on $I_i$ we know that it has a maximum value and a minimum value over each interval and these is well defined. \\
		Clearly $\underline{h} \leq f \leq \overline{h}$ and both $\underline{h}$ and $\overline{h}$ are p.c. functions.\\
		
		Next we will show that $\underline{\int}f = \overline{\int}f$ using an $\epsilon$ style argument. \\
		
		From the definition of p.c. integral it is easy to see that $$\text{pc}\int\underline{h} \leq \underline{\int}f \leq \overline{\int}f \leq \text{pc}\int \overline{h}$$ \\
		Let $\epsilon > 0$ be given. We would like to show that $\text{pc}\int\overline{h} - \text{pc}\int \underline{h} < \epsilon$.\\
		\begin{align*}
			\text{pc}\int\overline{h} - \text{pc}\int\underline{h} & = \sum\limits_{i=1}^n c_i m(I_i) - \sum\limits_{i=1}^n d_i m(I_i) \\
			& = \sum\limits_{i=1}^n (c_i - d_i) m(I_i)\\
			& = \sum\limits_{i=1}^n (c_i - d_i) \frac{b-a}{n}\\
			& < \sum\limits_{i=1}^n \frac{\epsilon}{b-a} \frac{b-a}{n} = \epsilon\\
		\end{align*}
		
		Note that in the above we can say that $(c_i - d_i) < \frac{\epsilon}{b-a}$ since $c_i,d_i \in I_i$ and so they are within a $\delta$ distance of eachother.\\
		
		By our epsilon argument, $f$ is Riemann integrable over $[a,b]$. \\
		
		We now prove that if $f$ is piecewise continuous and bounded, that it is Riemann integrable. \\
		
		Suppose that $f$ is continuous on each of a finite partition of intervals of $[a,b]$, and bounded. Let's say that there are $k$ intervals.\\
		On each of these intervals $I_j$ it is continuous. By the above argument we can define two piecewise constant functions on each interval $\underline{h}_j$ and $\overline{h}_j$ such that they bound $f$ over $I_j$ and $\text{pc}\int\overline{h}_j - \text{pc}\int \underline {h}_j < \frac{\epsilon}{k}$ of eachother. \\
		It is easy to see that $\underline{h}(x) = \underline{h}_j(x) \text{ for } x \in I_j$ is a p.c. function. \\
		Similarly $\overline{h}(x) = \overline{h}_j(x) \text{ for } x \in I_j$ is a p.c. function. \\
		Since any sub-interval of each $I_j$ is also a sub-interval of $[a,b]$, we can say $\underline{h} = \sum\limits_{j}\underline{h}_j$ and $\overline{h} = \sum\limits_{j}\overline{h}_j$. \\
		
		Then as above $\underline{h} \leq f \leq \overline{h}$, and we will show that $\underline{\int}f = \overline{\int}f$ using an $\epsilon$ style argument. \\
		
		Let $\epsilon > 0$ be given. We would like to show that $\text{pc}\int\overline{h} - \text{pc}\int \underline{h} < \epsilon$.\\
		\begin{align*}
			\text{pc}\int\underline{h} - \text{pc}\int\underline{h} & = \sum\limits_{j=1}^k \Big( \text{pc}\int\overline{h}_i \Big) - \sum\limits_{j=1}^k \Big( \text{pc}\int\underline{h}_i \Big) \\
			& = \sum\limits_{j=1}^k \Big( \text{pc}\int\overline{h}_i - \text{pc}\int\underline{h}_i \Big) \\
			& < \sum\limits_{j=1}^k \frac{\epsilon}{k} = \epsilon\\
		\end{align*}
		
		By our epsilon argument, $f$ is Riemann integrable over $[a,b]$. \\		
	\end{proof}



%problem 4.3
\hspace{-4 ex}\textbf{4.3} Show that the Riemann integral satisfies the linearity and monotonicity properties. \bigbreak

	\begin{proof}  \text{ }\\
		Linearity: \\
		
		Let $f,g$ be Riemann integrable over $[a,b]$, and let $c \in \mathbb{R}$. \\
		
		First note that if $c=0$ then $\int cf = \int 0 = 0$ since the zero function is a p.c. function. Then $0 \int f = \int 0 f$. (Note that we assume R.I. means $\int f$ is a finite real number.)\\
		Suppose $c > 0$. \\
		Let $\epsilon > 0$ be given. Then there exists p.c. functions $\underline{h}, \overline{h}$ such that $\underline{h} \leq f \leq \overline{h}$ and $\text{pc}\int \overline{h} - \underline{h} < \frac{\epsilon}{c}$. \\
		Then $c\underline{h}$, and $c\overline{h}$ are clearly p.c. functions where $c\underline{h} \leq cf \leq c\overline{h}$ and \\
		\begin{align*}
			\text{pc}\int c \overline{h} - \text{pc}\int c \underline{h} &  = c \Big(\text{pc}\int  \overline{h} - \text{pc}\int \underline{h} \Big) \text{\space \space \space since these are just sums}\\
			& < c \frac{\epsilon}{c} = \epsilon \\
		\end{align*}
		
		Now we have that\\
		\begin{align*}
			c *\text{pc}\int \underline{h} & \leq c \int f \leq c * \text{pc}\int \overline{h}  \\
			c *\text{pc}\int \underline{h} = \text{pc}\int c\underline{h} & \leq \int cf \leq \text{pc}\int c\overline{h} = c * \text{pc}\int \overline{h} \\
		\end{align*}
		
		Similarly to above $c \int f = \int cf$ if $c \geq 0$. \\
		
		Suppose $c < 0$. Then as above we have p.c. functions $\underline{h}, \overline{h}$ such that $\underline{h} \leq f \leq \overline{h}$, but this time $c\overline{h} \leq cf \leq c\underline{h}$ and  $\text{pc}\int \overline{h} - \underline{h} < -\frac{\epsilon}{c}$. \\
		Then \\
		$$\text{pc}\int c \underline{h} - \text{pc}\int c \overline{h}  = c \Big(\text{pc}\int  \underline{h} - \text{pc}\int \overline{h} \Big)$$ \\
		since these are just sums. Since $\text{pc}\int \overline{h} - \underline{h} < -\frac{\epsilon}{c}$, we know $\text{pc}\int \underline{h} - \overline{h} > \frac{\epsilon}{c}$, \\
		and $c \big( \text{pc}\int \underline{h} - \overline{h} \big) < \frac{\epsilon}{c}c = \epsilon$. \\
		
		Thus $c \int f = \int cf$. \\
		
		Next we show that $\int f +g = \int f + \int g$. \\
		
		Since $f,g$ are continuous, there exist p.c. functions $\underline{h}, \overline{h}, \underline{\phi}, \overline{\phi}$ such that \\
		\begin{align*}
			\underline{h} \leq f \leq \overline{h} &  &\underline{\phi} \leq g \leq \overline{\phi} \\
			\text{pc}\int \underline{h} - \overline{h} < \frac{\epsilon}{2} &  & \text{pc}\int \underline{\phi} - \overline{phi} < \frac{\epsilon}{2} \\
		\end{align*}
		
		Note that for any two p.c. functions (and hence any finite number) on the same interval, one can make a common refinement of their partitions and create equivalent p.c. functions over the same partition. Without loss of generality say that $\underline{h}, \overline{h}, \underline{\phi}, \overline{\phi}$ are p.c. functions over the same partition. Clearly, their sums are also p.c. functions, and the p.c. integrals of their sums are the sums of their p.c. integrals. \\
		
		Then  $\underline{h} + \underline{\phi} \leq f +g  \leq \overline{h} + \overline{\phi}$ and \\
		\begin{align*}
			\text{pc}\int \overline{h} + \overline{\phi} - \text{pc}\int  \underline{h} + \underline{\phi} & = \text{pc}\int \overline{h} - \underline{h} + \overline{\phi} - \underline{\phi} \\
			& = \text{pc}\int \overline{h} - \underline{h} + \text{pc}\int  \overline{\phi} + \underline{\phi} \\
			& < \frac{\epsilon}{2} +\frac{\epsilon}{2} = \epsilon
		\end{align*}
		
		Similarly to above $\int f +g = \int f + \int g$ and we have shown Riemann integration to be a linear operation.\\
		
		Monotonicity: \\
		
		Let $f,g$ be Riemann integrable and $f \leq g$. \\
		Then $0 \leq g-f$ and $g-f = g + (-1)f$ is Riemann integrable from above. \\
		Note that $0$ is a p.c. function. \\
		Then $0 = \text{pc}\int 0 \leq \int g-f = \int g - \int f$. \\
		So $\int f \leq \int g$. \\		
	\end{proof}



\end{document}
