\documentclass[12pt]{article}

%%%%%%%%%%%%%%%%%%%%%%%%%%%%%%%%%%%%%%%%%%%%%%%%%%%%%%%%%%%%%%%%%%%%%%%%%%%%%%%%%%%%%%%%%%%%%%%%%%%%
% Math
\usepackage{fancyhdr} 
\usepackage{amsfonts}
\usepackage{amsmath}
\usepackage{amssymb}
\usepackage{amsthm}
%\usepackage{dsfont}

%%%%%%%%%%%%%%%%%%%%%%%%%%%%%%%%%%%%%%%%%%%%%%%%%%%%%%%%%%%%%%%%%%%%%%%%%%%%%%%%%%%%%%%%%%%%%%%%%%%%
% Macros
\usepackage{calc}

%%%%%%%%%%%%%%%%%%%%%%%%%%%%%%%%%%%%%%%%%%%%%%%%%%%%%%%%%%%%%%%%%%%%%%%%%%%%%%%%%%%%%%%%%%%%%%%%%%%%
% Commands and Custom Variables	
\newcommand{\problem}[1]{\hspace{-4 ex} \large \textbf{Problem #1} }
\let\oldemptyset\emptyset
\let\emptyset\varnothing
\newcommand{\norm}[1]{\left\lVert#1\right\rVert}
\newcommand{\sint}{\text{s}\kern-5pt\int}
\newcommand{\powerset}{\mathcal{P}}
\renewenvironment{proof}{\hspace{-4 ex} \emph{Proof}:}{\qed}
\newcommand{\RR}{\mathbb{R}}
\newcommand{\NN}{\mathbb{N}}
\newcommand{\QQ}{\mathbb{Q}}
\newcommand{\ZZ}{\mathbb{Z}}
\newcommand{\CC}{\mathbb{C}}
\renewcommand{\Re}{\operatorname{Re}}
\renewcommand{\Im}{\operatorname{Im}}


%%%%%%%%%%%%%%%%%%%%%%%%%%%%%%%%%%%%%%%%%%%%%%%%%%%%%%%%%%%%%%%%%%%%%%%%%%%%%%%%%%%%%%%%%%%%%%%%%%%%
%page
\usepackage[margin=1in]{geometry}
\usepackage{setspace}
%\doublespacing
\allowdisplaybreaks
\pagestyle{fancy}
\fancyhf{}
\rhead{Shaw \space \thepage}
\setlength\parindent{0pt}

%%%%%%%%%%%%%%%%%%%%%%%%%%%%%%%%%%%%%%%%%%%%%%%%%%%%%%%%%%%%%%%%%%%%%%%%%%%%%%%%%%%%%%%%%%%%%%%%%%%%
%Code
\usepackage{listings}
\usepackage{courier}
\lstset{
	language=Python,
	showstringspaces=false,
	formfeed=newpage,
	tabsize=4,
	commentstyle=\itshape,
	basicstyle=\ttfamily,
}

%%%%%%%%%%%%%%%%%%%%%%%%%%%%%%%%%%%%%%%%%%%%%%%%%%%%%%%%%%%%%%%%%%%%%%%%%%%%%%%%%%%%%%%%%%%%%%%%%%%%
%Images
\usepackage{graphicx}
\graphicspath{ {images/} }
\usepackage{float}

%tikz
\usepackage[utf8]{inputenc}
\usepackage{pgfplots}
\usepgfplotslibrary{groupplots}

%%%%%%%%%%%%%%%%%%%%%%%%%%%%%%%%%%%%%%%%%%%%%%%%%%%%%%%%%%%%%%%%%%%%%%%%%%%%%%%%%%%%%%%%%%%%%%%%%%%%
%Hyperlinks
%\usepackage{hyperref}
%\hypersetup{
%	colorlinks=true,
%	linkcolor=blue,
%	filecolor=magenta,      
%	urlcolor=cyan,
%}

\begin{document}
	\thispagestyle{empty}
	
	\begin{flushright}
		Sage Shaw \\
		m515 - Fall 2017 \\
		\today
	\end{flushright}
	
{\large \textbf{HW 11}}\bigbreak
%%%%%%%%%%%%%%%%%%%%%%%%%%%%%%%%%%%%%%%%%%%%%%%%%%%%%%%%%%%%%%%%%%%%%%%%%%%%%%%%%%%%%%%%%%%%%%%%%%%%
\problem{Ex 21.3} Prove Holder's inequality for $p=1$ and $q=\infty$.

	\begin{proof}
		Let $f \in L^1$ and $g \in L^\infty$. By definition $g$ is bounded and $\norm{g}_\infty$ is the magnitude of the least upper bound of $g$. Then
		\begin{align*}
			\int \vert fg \vert & \leq \int \vert f \norm{g}_\infty \vert \\
			& = \norm{g}_\infty \int \vert f \vert \\
			& = \norm{g}_\infty \norm{f}_1
		\end{align*}
		Thus $fg$ is absolutely integrable and Holder's inequality holds.
	\end{proof}

	\bigbreak
Prove that $L^\infty$ is a Banach space with the norm $\norm{\cdot}_\infty$. \bigbreak

	\begin{proof}
		Let $f_n \in L^\infty$ be a sequence of functions that is Cauchy with respect to the norm $\norm{\cdot}_\infty$. First we will justify that $\{f_n\}$ converges pointwise and we will call the limit $f$. Let $\epsilon > 0$ be given and choose $N$ such that for any $n,m > N$ it is the case that $\norm{f_n - f_m}_\infty < \epsilon$. Let $x$ be in the domain. Then $\vert f_n(x) - f_m(x) \vert \leq \norm{f_n - f_m}_\infty < \epsilon$. So the sequence $\{f_n(x)\}$ is a Cauchy sequence of real numbers and is therefore convergent. Define $f(x) = \lim_{n \to \infty} f_n(x)$. Also, note that since $N$ is not dependent on $x$, this convergence is uniform.  \bigbreak
		
		Furthermore we will show that $\norm{f}_\infty$ is bounded. Let $\epsilon > 0$ be given and choose $N$ such that for any $n> N$ and any $x$, we have that $\vert f(x) - f_n(x)\vert < \epsilon$. Since this is true for all $x$ we can say that $\norm{f- f_n}_\infty \leq \epsilon$. Then
		\begin{align*}
			\norm{f}_\infty & = \norm{f - (f_N - f_N)}_\infty  \\
			& \leq \vert \norm{f - f_N}_\infty - \norm{f_N}_\infty \vert \\
			& \leq \epsilon + \norm{f_N}_\infty
		\end{align*}
		Thus Since $f_N \in L^\infty$ we have that $f$ is bounded and thus $f \in L^\infty$ and $L^\infty$ is complete. We conclude that $L^\infty$ is a Banach Space with the norm $\norm{\cdot}_\infty$. 
	\end{proof}

\bigbreak
%%%%%%%%%%%%%%%%%%%%%%%%%%%%%%%%%%%%%%%%%%%%%%%%%%%%%%%%%%%%%%%%%%%%%%%%%%%%%%%%%%%%%%%%%%%%%%%%%%%%
\problem{Ex 22.1 (BBT, ex 13:6.1)} Let $g\in L^1[0,1]$. Show that the map $f\mapsto\int fg$ is a bounded linear functional on $L^\infty[0,1]$.

	\begin{proof}
		Since integration is linear this is obviously a linear functional. Let $f \in L^\infty$. Then
		\begin{align*}
			\left \vert \int fg \right \vert & \leq \left \vert \int \norm{f}_\infty g \right \vert \\
			& = \norm{f}_\infty \left \vert \int g \right \vert  \\
			& \leq \norm{f}_\infty \int \vert g \vert \\
			& = \norm{f}_\infty \norm{g}_1
		\end{align*}
		Thus our functional is bounded by $\norm{g}_1$.
	\end{proof}

\bigbreak
%%%%%%%%%%%%%%%%%%%%%%%%%%%%%%%%%%%%%%%%%%%%%%%%%%%%%%%%%%%%%%%%%%%%%%%%%%%%%%%%%%%%%%%%%%%%%%%%%%%%
\problem{Ex 22.2 (BBT, ex 13:6.2).} Show that there is a nonzero bounded linear functional on $L^\infty[0,1]$ that vanishes on the (closed) subspace of continuous functions. \bigbreak

	\begin{proof}
		The functional on $L^\infty [0, 1]$ will be the distance of the step discontinuity at the point $\tfrac{1}{2}$ from left to right. \\
		Formally we will define the functional $\phi: L^\infty [0, 1] \to \RR$ by
		$$
		\phi(f) = \inf_{\delta > 0} \Big \{ \sup_{x \in (.5, .5 + \delta)} \{ f(\tfrac{1}{2}) - f(x) \} \Big \}
		$$
		For continuous functions (in particular they are continuous at $\tfrac{1}{2}$) this value is zero, by definition. Since $f$ is bounded, the largest that $\phi$ could scale it, is to double it (if the function jumped up from the negative bound to the positive bound at $\tfrac{1}{2}$). Clearly $\phi(cf) = c\phi(f)$. \\
		$\phi(f + g) = \phi(f) + \phi(g)$ is harder to justify formaly, but is true. \\
		Lastly note that $\phi(\chi_{(.5,1)}) = 1$ so $\phi$ is non-zero.
	\end{proof}

\bigbreak
%%%%%%%%%%%%%%%%%%%%%%%%%%%%%%%%%%%%%%%%%%%%%%%%%%%%%%%%%%%%%%%%%%%%%%%%%%%%%%%%%%%%%%%%%%%%%%%%%%%%
\problem{Ex 22.3 (BBT, ex 13:6.3)} Show that there is a bounded linear functional on $L^\infty[0,1]$ that is not of the form $f\mapsto\int fg$ for any $g\in L^1[0,1]$.

\end{document}
