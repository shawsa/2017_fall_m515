\documentclass[12pt]{article}


% Math		****************************************************************************************
\usepackage{fancyhdr} 
\usepackage{amsfonts}
\usepackage{amsmath}
\usepackage{amssymb}
\usepackage{amsthm}
%\usepackage{dsfont}

% Commands and Custom Variables	********************************************************************
\newcommand{\problem}[1]{\hspace{-4 ex} \large \textbf{#1}\\}
\let\oldemptyset\emptyset
\let\emptyset\varnothing

%page		****************************************************************************************
\usepackage[margin=1in]{geometry}
\usepackage{setspace}
\doublespacing
\pagestyle{fancy}
\fancyhf{}
\rhead{Shaw \space \thepage}
\setlength\parindent{0pt}


\begin{document}
	\thispagestyle{empty}
	
	\begin{flushright}
		Sage Shaw \\
		m515 - Fall 2017 \\
		\today
	\end{flushright}
	
\large{\textbf{HW 4: 7.1, 7.2, 7.3, 8.1, 8.2}}\\

\problem{7.1 (a)} Show that $A$ is measurable iff for all $\epsilon>0$ there exists a closed set $F\subset A$ such that $m^*(A\setminus F)<\epsilon$.

	\begin{proof}
		Let $A$ be Lebesgue measurable and let $\epsilon > 0$ be given. \\
		Since $A$ is measurable, $A^c$ is measurable, and there exisists an open set $O$ such that $A^c \subseteq O$ and $m^*(O \setminus A^c) < \epsilon$. Then 
		\begin{align*}
			m^*(A \setminus O^c) & = m^*(A \cap (O^c)^c) \\
			& = m^*(A \cap O) \\
			& = m^*(O \setminus A^c) \\
			& < \epsilon
		\end{align*}
		Since $O^c$ is closed, we've obtained our desired result. \bigbreak
		
		Now suppose that for all $\epsilon > 0$ there exist a closed set $F$ such that $F \subseteq A$ and $m^*(A \setminus F) < \epsilon$. \\
		Let $\epsilon >0$ be given. Choose a closed set $F$ such that $F \subseteq A$ and $m^*(A \setminus F) < \epsilon$. Then
		\begin{align*}
			m^*(F^c \setminus A^c) & = m^*(F^c \cap A) \\
			& = m^*(A \setminus F) \\
			& < \epsilon
		\end{align*}
		Since $F^c$ is open, we can say that $A^c$ is Lebesgue measurable. Then $(A^c)^c = A$ is Lebesgue measurable as well.		
	\end{proof}

\problem{7.1 (b)} Show that $A$ is measurable iff for all $\epsilon>0$ there exists a measurable set $B$ such that $m^*(A\triangle B)<\epsilon$.

	\begin{proof}
		The first direction is trivial: if $A$ is measurable, then $A$ is a measurable set such that $m^*(A \setminus A) = m^*(\emptyset) < \epsilon$.
	\end{proof}

\problem{7.2} Show that any set $A$ is contained in a Lebesgue measurable set $B$ such that $m(B)=m^*(A)$.

	\begin{proof}
		First if $m^*(A) = \infty$, note that $A \subseteq \mathbb{R}^n$ and $m^*(A) = \infty = m^*(\mathbb{R}^n)$. Now suppose that $m^*(A)$ is finite. \\
		Let $A$ be a set. From Lemma 6.5 it follows that for any $n \in \mathbb{N}$ there exisits an open set $O_n^\prime$ such that $m^*(O_n^\prime) - m^*(A) < \tfrac{1}{n}$ and $A \subseteq O_n^\prime$. Define $O_1 = O_1^\prime$, and $O_n = O_{n}^\prime \cap O_{n-1}$. Clearly, for each $n$, $O_{n+1} \subseteq O_n$, $A \subseteq O_n$, $m^*(O_n)$ is finite and $O_n$ is measurable. Using the Downward Monontone Convergence theorem we obtain the following
		\begin{align*}
			m^*(\bigcap O_n) & = \lim_{n \to \infty} m^*(O_n) \\
			& \leq \lim_{n \to \infty} m^*(A) + \tfrac{1}{n} \\
			& = m^*(A)
		\end{align*}
		Since each $O_n$ is open (finite intersection of open sets) $\bigcap O_n$ is measurable and $m(\bigcap O_n) = m^*(A)$.
	\end{proof}

\problem{7.3} Show the \emph{inner regularity} property: If $A$ is Lebesgue measurable, then $m(A)=\text{sup}\{m(K)\mid K\subset A, K\text{ compact}\}$.

\problem{8.1} Give a counterexample showing that the hypothesis that some $A_n$ has finite measure is necessary for the downwards MCT.

	\begin{proof}
		Consider the sequence of boxes $B_n = [0,\tfrac{1}{n}] \times [0, \infty)$. Clearly $B_{n+1} \subseteq B_n$ and $m(B_n)=\infty$ for each $n \in \mathbb{N}$. Thus $\lim_{n \to \infty}m(B_n) = \infty$ as well. However $\bigcap_{n=1}^{\infty}B_n = [0,0] \times [0,\infty)$ and has measure $0$.
	\end{proof}

\problem{8.2} Suppose you know that $m$ satisfies $m(\emptyset)=0$ and the countable additivity property. Show that $m$ satisfies the monotonicity property and the countable subadditivity property (for measurable sets). (Morgan says that Sam said that we can assume the measurable sets form a sigma algebra.)


\end{document}
