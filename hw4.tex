\documentclass[12pt]{article}


% Math		****************************************************************************************
\usepackage{fancyhdr} 
\usepackage{amsfonts}
\usepackage{amsmath}
\usepackage{amsthm}
%\usepackage{dsfont}

% Commands	****************************************************************************************
\newcommand{\problem}[1]{\hspace{-4 ex} \large \textbf{#1}\\}

%page		****************************************************************************************
\usepackage[margin=1in]{geometry}
\usepackage{setspace}
\doublespacing
\pagestyle{fancy}
\fancyhf{}
\rhead{Shaw \space \thepage}
\setlength\parindent{0pt}


\begin{document}
	\thispagestyle{empty}
	
	\begin{flushright}
		Sage Shaw \\
		m515 - Fall 2017 \\
		\today
	\end{flushright}
	
\large{\textbf{HW 4: 7.1, 7.2, 7.3, 8.1, 8.2}}\\

\problem{7.1 (a)} Show that $A$ is measurable iff for all $\epsilon>0$ there exists a closed set $F\subset A$ such that $m^*(A\setminus F)<\epsilon$.

\problem{7.1 (b)} Show that $A$ is measurable iff for all $\epsilon>0$ there exists a measurable set $B$ such that $m^*(A\triangle B)<\epsilon$.

\problem{7.2} Show that any set $A$ is contained in a Lebesgue measurable set $B$ such that $m(B)=m^*(A)$.

\problem{7.3} Show the \emph{inner regularity} property: If $A$ is Lebesgue measurable, then $m(A)=\text{sup}\{m(K)\mid K\subset A, K\text{ compact}\}$.

\problem{8.1} Give a counterexample showing that the hypothesis that some $A_n$ has finite measure is necessary for the downwards MCT.

\problem{8.2} Suppose you know that $m$ satisfies $m(\emptyset)=0$ and the countable additivity property. Show that $m$ satisfies the monotonicity property and the countable subadditivity property (for measurable sets).


\end{document}
