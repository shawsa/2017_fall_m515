\documentclass[12pt]{article}


% Math		****************************************************************************************
\usepackage{fancyhdr} 
\usepackage{amsfonts}
\usepackage{amsmath}
\usepackage{amssymb}
\usepackage{amsthm}
%\usepackage{dsfont}

% Commands and Custom Variables	********************************************************************
\newcommand{\problem}[1]{\hspace{-4 ex} \large \textbf{#1}\\}
\let\oldemptyset\emptyset
\let\emptyset\varnothing

%page		****************************************************************************************
\usepackage[margin=1in]{geometry}
\usepackage{setspace}
\doublespacing
\pagestyle{fancy}
\fancyhf{}
\rhead{Shaw \space \thepage}
\setlength\parindent{0pt}


\begin{document}
	\thispagestyle{empty}
	
	\begin{flushright}
		Sage Shaw \\
		m515 - Fall 2017 \\
		\today
	\end{flushright}
	
\large{\textbf{HW 4: 7.1, 7.2, 7.3, 8.1, 8.2}}\\

\problem{7.1 (a)} Show that $A$ is measurable iff for all $\epsilon>0$ there exists a closed set $F\subseteq A$ such that $m^*(A\setminus F)<\epsilon$.

	\begin{proof}
		Let $A$ be Lebesgue measurable and let $\epsilon > 0$ be given. \\
		Since $A$ is measurable, $A^c$ is measurable, and there exisists an open set $O$ such that $A^c \subseteq O$ and $m^*(O \setminus A^c) < \epsilon$. Then 
		\begin{align*}
			m^*(A \setminus O^c) & = m^*(A \cap (O^c)^c) \\
			& = m^*(A \cap O) \\
			& = m^*(O \setminus A^c) \\
			& < \epsilon
		\end{align*}
		Since $O^c$ is closed, we've obtained our desired result. \bigbreak
		
		Now suppose that for all $\epsilon > 0$ there exist a closed set $F$ such that $F \subseteq A$ and $m^*(A \setminus F) < \epsilon$. \\
		Let $\epsilon >0$ be given. Choose a closed set $F$ such that $F \subseteq A$ and $m^*(A \setminus F) < \epsilon$. Then
		\begin{align*}
			m^*(F^c \setminus A^c) & = m^*(F^c \cap A) \\
			& = m^*(A \setminus F) \\
			& < \epsilon
		\end{align*}
		Since $F^c$ is open, we can say that $A^c$ is Lebesgue measurable. Then $(A^c)^c = A$ is Lebesgue measurable as well.		
	\end{proof}

\problem{7.1 (b)} Show that $A$ is measurable iff for all $\epsilon>0$ there exists a measurable set $B$ such that $m^*(A\triangle B)<\epsilon$.

	\begin{proof}
		The first direction is trivial: if $A$ is measurable, then $A$ is a measurable set such that $m^*(A \setminus A) = m^*(\emptyset) = 0 < \epsilon$. \bigbreak
		Let $\epsilon > 0$ be given. From our premise there exists a measurable $B$ such that $m^*(A \Delta B) < \tfrac{\epsilon}{2}$. Clearly then, $m^*(A \setminus B) < \tfrac{\epsilon}{2}$. Since $B$ is measurable, there exists a closed set $F \subseteq B$ such that $m^*(B \setminus F) \tfrac{\epsilon}{2}$. Finally,
		\begin{align*}
			m^*(A \setminus F) & = m^*\Big((A \setminus B) \cup (B \setminus F)\Big) \\
			& = m^*(A \setminus B) + m^*(B \setminus F) \\
			& < \tfrac{\epsilon}{2} + \tfrac{\epsilon}{2} \\
			& = \epsilon
		\end{align*}
		Thus by part a, $A$ is measurable.
	\end{proof}

\problem{7.2} Show that any set $A$ is contained in a Lebesgue measurable set $B$ such that $m(B)=m^*(A)$.

	\begin{proof}
		First if $m^*(A) = \infty$, note that $A \subseteq \mathbb{R}^n$ and $m^*(A) = \infty = m^*(\mathbb{R}^n)$. Now suppose that $m^*(A)$ is finite. \\
		Let $A$ be a set. From Lemma 6.5 it follows that for any $n \in \mathbb{N}$ there exisits an open set $O_n^\prime$ such that $m^*(O_n^\prime) - m^*(A) < \tfrac{1}{n}$ and $A \subseteq O_n^\prime$. Define $O_1 = O_1^\prime$, and $O_n = O_{n}^\prime \cap O_{n-1}$. Clearly, for each $n$, $O_{n+1} \subseteq O_n$, $A \subseteq O_n$, $m^*(O_n)$ is finite and $O_n$ is measurable. Using the Downward Monontone Convergence theorem we obtain the following
		\begin{align*}
			m^*\Big(\bigcap O_n \Big) & = \lim_{n \to \infty} m^*(O_n) \\
			& \leq \lim_{n \to \infty} m^*(O_n^\prime) \\
			& \leq \lim_{n \to \infty} m^*(A) + \tfrac{1}{n} \\
			& = m^*(A)
		\end{align*}
		Since each $O_n$ is open (finite intersection of open sets) $\bigcap O_n$ is measurable and $m(\bigcap O_n) = m^*(A)$.
	\end{proof}

\problem{7.3} Show the \emph{inner regularity} property: If $A$ is Lebesgue measurable, then $m(A)=\text{sup}\{m(K)\mid K\subseteq A, K\text{ compact}\}$.

%	\begin{proof}
%		Clearly $m(A)$ is an upper bound on $\{m(K)\mid K\subseteq A, K\text{ compact}\}$ from monotonicity. Let $\epsilon >0$ be given. It remains to show that there exists a compact set $K$, such that $K \subseteq A$ and $m(A \setminus K) < \epsilon$. \bigbreak
%		Suppose that $m(A)$ is bounded. Then from problem 7.1(a) we know that there exists a closed set $K$ such that $K \subseteq A$ and $m(A \setminus K) < \epsilon$. Since $K$ is closed and bounded it is compact. \bigbreak
%		
%		Suppose that $A$ is not bounded ...
%	\end{proof}
	
	\begin{proof}
		Clearly $\{m(K)\mid K\subseteq A, K\text{ compact}\}$ is bounded above by $m(A)$. \\
		It remains to show that for any $\epsilon > 0$, there exists a compact set $K$, such that $m(A) - m(K) < \epsilon$. Let $\epsilon > 0$ be given.
		By the definition of outer Lebesgue measure there exists $\{B_n\}_{n=1}^\infty$ such that $A \subseteq \bigcup B_n$. \\% and $\sum\limits_{n=1}^\infty \text{vol}(B_n) = m(A) + \epsilon$.\\
		Define $C_n = A \cap \bigcup\limits_{i=1}^n B_i = \bigcup\limits_{i=1}^n(A \cap B_i)$. Clearly $\bigcup\limits_{n=1}^\infty C_n = A$, $C_n \subseteq A$, each $C_n$ is bounded and $C_n \subseteq C_{n+1}$ for any $n \in \mathbb{N}$. So $\lim\limits_{n \to \infty}m(C_n) = m(A)$. By definition of limit, there exists an $N \in \mathbb{N}$ such that $m(A) - m(C_N) < \tfrac{\epsilon}{2}$. From problem 7.1(a) we know that there exists a closed set $F \subseteq C_N$ such that $m(C_N \setminus F) < \tfrac{\epsilon}{2}$. Since $C_N$ is bounded, $F$ is bounded, and is thus compact. Finally since $F \subseteq C_N \subseteq A$,
		\begin{align*}
			m(A) - m(F) & = m(A) - m(C_N) + m(C_N) -  m(F) \\
			& = m(A \setminus C_N) + m(C_N \setminus F) \\
			& < \tfrac{\epsilon}{2} + \tfrac{\epsilon}{2} \\
			& = \epsilon
		\end{align*}
	\end{proof}

\problem{8.1} Give a counterexample showing that the hypothesis that some $A_n$ has finite measure is necessary for the downwards MCT.

	\begin{proof}
		Consider the sequence of boxes $B_n = [0,\tfrac{1}{n}] \times [0, \infty)$. Clearly $B_{n+1} \subseteq B_n$ and $m(B_n)=\infty$ for each $n \in \mathbb{N}$. Thus $\lim\limits_{n \to \infty}m(B_n) = \infty$ as well. However $\bigcap\limits_{n=1}^{\infty}B_n = [0,0] \times [0,\infty)$ and has measure $0$.
	\end{proof}

\problem{8.2} Suppose you know that $m$ satisfies $m(\emptyset)=0$ and the countable additivity property. Show that $m$ satisfies the monotonicity property and the countable subadditivity property (for measurable sets). \textbf{Note:} Tao's version of this problem give us that $m$ maps all Lebesgue measurable sets into $[0,\infty]$. Since the Lebesgue measurable sets form a $\sigma$-algebra, the measurable sets by our measure will too.

	\begin{proof}
		Let $A$ and $B$ be sets such that $A \subseteq B$. By our assumption, $B \setminus A$ is measurable. Then
		\begin{align*}
			m(A) & \leq m(A) + m(B \setminus A) \text{\space \space (measure is positive)}\\
			& = m(A) + m(B \setminus A) + \sum_{n=3}^\infty m(\emptyset) \\
			& = m\Big(A \cup (B \setminus A) \cup \bigcup_{n=3}^\infty \emptyset\Big) \text{\space \space (disjoint)}\\
			& = m(A \cup B) \\
			& = m(B) \text{\space \space (subset)}
		\end{align*}
		Thus we have monotonicity of $m$. \bigbreak
		Now suppose we have sets $A_n$ for $n \in \mathbb{N}$. \\
		Let $C_1 = A_1$ and $C_n = A_n \setminus \bigcup\limits_{i =1}^{n-1}C_i$ for $n>1$. Intuitively $C_n$ is now the elements of $A_n$ that are in no other $A_i$ for $i<n$. We know that each $C_n$ is measurable by our assumption. Clearly $\bigcup C_n = \bigcup A_n$ and our $C_n$ are pairwise disjoint (see lemma below, or don't). Since $C_n \subseteq A_n$, we know that $m(C_n) \leq m(A_n)$. Then 
		\begin{align*}
			m \Big (\bigcup A_n \Big) & = m \Big( \bigcup C_n \Big) \\
			& = \sum m(C_n) \text{\space \space (countable addititivity)} \\
			& \leq \sum m(A_n)			
		\end{align*}
	\end{proof}
	\bigbreak
	\bigbreak
	
	Since Daniel and Jordon are going to bother to prove that $\bigcup C_n = \bigcup A_n$ and also $\{C_n\}$ are pairwise disjoint, I might as well do it too...
	
\problem{Aforementioned Lemma...}
	Given a countable collection of sets $\{A_n\}$. Define $C_1 = A_1$ and \\
	$C_n = A_n \setminus \bigcup\limits_{i =1}^{n-1}C_i$ for $n>1$. Then our $C_n$ are pairwise disjoint and \\
	$\bigcup C_n = \bigcup A_n$.
	
	\begin{proof}
		Given $A_n$ and $C_n$ as defined above, let's show $\bigcup C_n = \bigcup A_n$. Clearly $\bigcup C_n \subseteq \bigcup A_n$ since $C_n \subseteq A_n$ for each $n \in \mathbb{N}$. Let $x \in \bigcup A_n$. Then $x \in A_k$ for some $k \in \mathbb{N}$. Since $C_n = A_n \setminus \bigcup\limits_{i =1}^{n-1}C_i$ either $x \in C_k$, or $x \in \bigcup\limits_{i =1}^{k-1}C_i$. Thus $x \in \bigcup C_n$. \bigbreak
		Now we will prove that each $C_n$ is disjoint. Since $C_2 = A_2 \setminus C_1$, we know that $C_1$ and $C_2$ are disjoint. Now assume that for some $k \in \mathbb{N}$, $C_j$ and $C_k$ are disjoint for all $j \in \{1,2,...,k-1\}$. Since $C_{k+1} = A_{k+1} \ \bigcup\limits_{i =1}^{k}C_i$ we know that $C_j$ and $C_{k+1}$ are disjoint for all $j \in \{1,2,...,k\}$. Thus our $C_n$ are pairwise disjoint by induction.
	\end{proof}

\end{document}
