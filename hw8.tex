\documentclass[12pt]{article}

%***************************************************************************************************
% Math
\usepackage{fancyhdr} 
\usepackage{amsfonts}
\usepackage{amsmath}
\usepackage{amssymb}
\usepackage{amsthm}
%\usepackage{dsfont}

%***************************************************************************************************
% Macros
\usepackage{calc}

%***************************************************************************************************
% Commands and Custom Variables	
\newcommand{\problem}[1]{\hspace{-4 ex} \large \textbf{Problem #1} }
\let\oldemptyset\emptyset
\let\emptyset\varnothing
\newcommand{\norm}[1]{\left\lVert#1\right\rVert}
\newcommand{\sint}{\text{s}\kern-5pt\int}
\newcommand{\powerset}{\mathcal{P}}
\renewenvironment{proof}{\hspace{-4 ex} \emph{Proof}:}{\qed}
\newcommand{\RR}{\mathbb{R}}
\newcommand{\NN}{\mathbb{N}}
\newcommand{\QQ}{\mathbb{Q}}
\newcommand{\ZZ}{\mathbb{Z}}
\newcommand{\CC}{\mathbb{C}}
\newcommand{\BB}{\mathcal{B}}
\newcommand{\SA}{\mathcal{A}}



%***************************************************************************************************
%page
\usepackage[margin=1in]{geometry}
\usepackage{setspace}
%\doublespacing
\allowdisplaybreaks
\pagestyle{fancy}
\fancyhf{}
\rhead{Shaw \space \thepage}
\setlength\parindent{0pt}

%***************************************************************************************************
%Code
\usepackage{listings}
\usepackage{courier}
\lstset{
	language=Python,
	showstringspaces=false,
	formfeed=newpage,
	tabsize=4,
	commentstyle=\itshape,
	basicstyle=\ttfamily,
}

%***************************************************************************************************
%Images
\usepackage{graphicx}
\graphicspath{ {images/} }
\usepackage{float}

%tikz
\usepackage[utf8]{inputenc}
\usepackage{pgfplots}
\usepgfplotslibrary{groupplots}

%***************************************************************************************************
%Hyperlinks
%\usepackage{hyperref}
%\hypersetup{
%	colorlinks=true,
%	linkcolor=blue,
%	filecolor=magenta,      
%	urlcolor=cyan,
%}

\begin{document}
	\thispagestyle{empty}
	
	\begin{flushright}
		Sage Shaw \\
		m515 - Fall 2017 \\
		\today
	\end{flushright}
	
{\large \textbf{HW 8}}\bigbreak

%%%%%%%%%%%%%%%%%%%%%%%%%%%%%%%%%%%%%%%%%%%%%%%%%%%%%%%%%%%%%%%%%%%%%%%%%%%%%%%%%%%%%%%%%%%%%%%%%%%%
\problem{15.1} 
Let $(X, \SA,\mu_0)$ be a premeasure and let $(X,\mathcal B,\mu)$ be the Carath\'eodory extension. Show that for any $B \in \BB$ there exists $C$ in the $\sigma$-algebra generated by $\ \SA$ such that $B\subset C$ and $\mu(C\setminus B)=0$.
\bigbreak

\begin{proof}
	Let $B \in \BB$. Let $\langle \SA \rangle$ denote the $\sigma$-algebra generated by $\SA$. By definition $\mu(B) = \inf \{ \sum \mu_0(A_i) \mid A_i \in \mathcal{A}, \text{ and } B \subseteq \bigcup A_i \}$. Assume momentarily that $\mu(B)$ is finite. So for every $n \in \NN$, there exists a countable collection of sets $C_{n,i} \in \mathcal{A}$ such that $\sum\limits_{i \in \NN} \mu_0(C_{n,i}) - \mu(B) < \tfrac{1}{n}$ and $B \subseteq \bigcup\limits_{i \in \NN} C_{n,i}$. Let $C_n = \bigcup\limits_{i \in \NN} C_{n,i}$. Then $\mu(C_n) - \mu(B) < \tfrac{1}{n}$ and $C_n \in \langle \SA \rangle$. Since $B \subseteq C_n$, $\mu(C_n \setminus B) < \tfrac{1}{n}$. Define $K_n = \bigcap_{i=1}^n C_i$, and define $K = \bigcap_{i=1}^\infty C_i$ (both clearly in $\langle \SA \rangle$). Then $B \subseteq K_n$ for all $n \in \NN$ and $B \subseteq K$. Also $K_{n+1} \subseteq K_n$. Clearly $\mu(K_1)$ is finite. Since $\mu(K_n) < \tfrac{1}{n} + \mu(B)$, by DMCT we have that $\mu(K) \leq \mu(B)$ and from monotonicity we have that $\mu(K) = \mu(B)$. Then since $B \subseteq K$, $\mu(K \setminus B) = 0$. \bigbreak
	
	\textbf{************** prove for infinite measure case *****************}
	Is this the analogue to the theorem that every Lebesgue measureable set can be approximated by a Borel set?
\end{proof}



\bigbreak
%%%%%%%%%%%%%%%%%%%%%%%%%%%%%%%%%%%%%%%%%%%%%%%%%%%%%%%%%%%%%%%%%%%%%%%%%%%%%%%%%%%%%%%%%%%%%%%%%%%%
\problem{15.2} Let $(X,\mathcal A,\mu_0)$ be a premeasure, $\mu^*$ the corresponding outer measure, and $(X,\mathcal B,\mu)$ the Carath\'eodory extension. Show that for any set $B$ such that $\mu^*(B)<\infty$, we have $B\in\mathcal B$ if and only if for all $\epsilon>0$ there exists $A\in\mathcal A$ such that $\mu^*(A\triangle B)<\epsilon$.
\bigbreak

\begin{proof}
	
	$\Longrightarrow$
	
	Let $B \in \BB$ and let $\epsilon > 0$ be given. Then there exists a collection of $A_n \in \SA$ such that $B \subseteq \bigcup A_n$ and $\sum \mu_0(A_n) - \mu(B) < \tfrac{\epsilon}{2}$. Then clearly $\sum \mu_0(A_n)$ is finite and converges. Thus there exists some $N \in \NN$ such that $\sum\limits_{n=1}^\infty \mu_0(A_n) - \sum\limits_{n=1}^N \mu_0{A_n} < \tfrac{\epsilon}{2}$. Let $A = \bigcup\limits_{n=1}^N A_n$. Then $A \in \SA$ and 
	\begin{align*}
		\mu^*(A \bigtriangleup B) 
		& = \mu^* \big( (A \setminus B) \cup (B \setminus A) \big ) \\
		& = \mu^*(A \setminus B) + \mu^*(B \setminus A) \\
		& \leq  \mu^* \Big( \bigcup A_n \setminus B \Big ) + \mu^* \Big( \bigcup A_n \setminus A \Big ) \\
		& < \tfrac{\epsilon}{2} + \tfrac{\epsilon}{2}
	\end{align*}
	
	$\Longleftarrow$
	
	Let $B \subseteq X$ such that $\mu^*(B) < \infty$. Suppose that for each $\epsilon > 0$ there exists an $A \in \SA$ such that $\mu^*(A \bigtriangleup B) < \epsilon$. Let $S \subseteq X$. From subadditivity we have that 
	$$
	\mu^*(S) \leq \mu^*(S \cap B) + \mu(S \cap B^*)
	$$
	From set theory we know that
	$$
	S \cap B = (S \cap B \cap A) \cup (S \cap B \cap A^c)
	$$$$
	S \cap B^c = (S \cap B^c \cap A) \cup (S \cap B^c \cap A^c)
	$$
	Since $A$ is measurable we have that 
	\begin{align*}
		\mu^*(S \cap B) & = \mu^*(S \cap B \cap A) + \mu^*(S \cap B \cap A^c) \\
		\mu^*(S \cap B^c) & = \mu^*(S \cap B^c \cap A) + \mu^*(S \cap B^c \cap A^c)
	\end{align*}
	Again from set theory we know that $A \bigtriangleup B = A^c \bigtriangleup B^c$ and so by monotonicity we have
	\begin{align*}
		\mu^*(S \cap B) & \leq \mu^*(S \cap B \cap A) + \epsilon \\
		\mu^*(S \cap B^c) & \leq \epsilon + \mu^*(S \cap B^c \cap A^c)
	\end{align*}
	Then
	\begin{align*}
		\mu^*(S \cap B) + \mu^*(S \cap B^c) & \leq 2\epsilon + \mu^*(S \cap B \cap A) + \mu^*(S \cap B^c \cap A^c) \\
		& \leq 2\epsilon + \mu^*(S \cap A) + \mu^*(S \cap A^c) \text{    (by monotonicity)} \\
		& = 2\epsilon + \mu^*(S) \text{ (since $A$ is measurable)}
	\end{align*}
	Now we have that for all $\epsilon$
	$$
	\mu^*(S) \leq \mu^*(S \cap B) + \mu(S \cap B^*) \leq 2\epsilon + \mu^*(S)
	$$
	Taking $\epsilon$ to zero we have that $\mu^*(S) = \mu^*(S \cap B) + \mu(S \cap B^*)$ and $B$ is measurable.	
\end{proof}


\bigbreak
%%%%%%%%%%%%%%%%%%%%%%%%%%%%%%%%%%%%%%%%%%%%%%%%%%%%%%%%%%%%%%%%%%%%%%%%%%%%%%%%%%%%%%%%%%%%%%%%%%%%
\problem{16.1} Show that $f$ is measurable if and only if $f^+$ and $f^-$ are measurable.
\bigbreak

\begin{proof}
	Let $(X, \BB,\mu)$ be a measure space.
	
	$\Longrightarrow$
	
	Let $f: X \to \RR$ be a $\mu$-measurable function and let $g = f^+$ (the notation $f^{+-1}$ just seems silly to me). Let $O \subseteq \RR$ be an open set. Then 
	\begin{align*}
		g^{-1}(O) & = g^{-1}\Big( \big( O \setminus (-\infty,0) \big) \cup \big( O \setminus [0, \infty) \big) \Big) \\
		& = g^{-1} \big( O \setminus (-\infty,0) \big) \cup g^{-1} \big( O \setminus [0, \infty) \big) \\
		& = \emptyset \cup f^{-1} \big( O \setminus [0, \infty) \big) \\
		& = f^{-1} \big( O \setminus [0, \infty) \big) \\
	\end{align*}
	To prove $f^+$ is $\mu$-measurable it remains to show that $f^{-1} \big( O \setminus [0, \infty) \big)$ is $\mu$-measurable.
	Since $(0, \infty),$ and $(-\infty, 0)$ are open and $\BB$ is a $\sigma$-algebra
	$$
	f^{-1} \big( (0, \infty) \cup (-\infty, 0) \big)^C = f^{-1} \big( \{0\} \big ) \in \BB
	$$
	$$
	f^{-1} \big( (0, \infty) \big) \cup f^{-1} \big( \{0\} \big ) = f^{-1} \big( [0, \infty) \big) \in \BB
	$$
	Finally $f^{-1} \big( O \setminus [0, \infty) \big) = f^{-1}(O) \setminus f^{-1} \big( [0, \infty) \big) \in \BB$ since $\BB$ is a $\sigma$-algebra. A similar argument works for $f^-$.
	
	\bigbreak $\Longleftarrow$
	
	Let $f: X \to \RR$ be a function, let $g = f^+$ and $h= f^-$, and suppose that both $g$ and $h$ are measurable. Let $O \in \RR$ be an open set. Then $O = O^+ \cup O^- \cup O^0$ where
	\begin{align*}
		O^+ & = O \cap (0,\infty) \\
		O^- & = O \cap (-\infty, 0) \\
		O^0 & = O \cap \{0\}
	\end{align*}
	Note that $O^+$ and $O^-$ are open since openness is preserved under pairwise intersections.\\
	Define $-O^- = \{-x \mid x\in O\}$. Since the function $n(x) = -x$ is continuous, $-O^- = n^{-1}(O^-)$ is open. Then
	\begin{align*}
		f^{-1}(O) & = f^{-1}(O^+) \cup f^{-1}(O^-) \cup f^{-1}(O^0) \\
		& = g^{-1}(O^+) \cup h^{-1}(-O^-) \cup f^{-1}(O^0) \\
	\end{align*}
	Since $g$ and $h$ are $\mu$-measurable, $g^{-1}(O^+) \cup h^{-1}(-O^-) \in \BB$. It remains to show that $f^{-1}(O^0)$ is $\mu$-measurable. To simplify matters, notice that either $O^0 = \emptyset$ or $O^0 = \{0\}$. Clearly $f^{-1}(\emptyset) = \emptyset \in \BB$. Notice that 
	$$
	f^{-1}\big ( \{0\} \big ) = g^{-1}\big ( \{0\} \big ) \cap h^{-1}\big ( \{0\} \big )
	$$
	From above we know that $g^{-1}\big ( \{0\} \big ), h^{-1}\big ( \{0\} \big ) \in \BB$ and thus $f^{-1}(O^0)$ is $\mu$-measurable and finally $f$ is $\mu$-measurable.	
\end{proof}

\bigbreak

Show that sums and products of measurable functions are measurable. \bigbreak

\begin{proof}
	Let $f,g: X \to \RR$ be $\mu$-measurable functions. First let us consider the open ray $(-\infty,t)$. It is the case that 
	$$
	(f+g)^{-1}\big ( (-\infty,t) \big) = \bigcup_{r \in \QQ} \Big[ f^{-1}\big( (-\infty,r) \big) \cap g^{-1}\big( (-\infty,t-r) \big) \Big]
	$$
	This is not obvious so we will prove it. \bigbreak
	
	For notational convenience let
	$$
	L = (f+g)^{-1}\big ( (-\infty,t) \big) \text{\ \ \ \ (left side)}
	$$
	and 
	$$
	R = \bigcup_{r \in \QQ} \Big[ f^{-1}\big( (-\infty,r) \big) \cap g^{-1}\big( (-\infty,t-r) \big) \Big] \text{\ \ \ \ \ (right side)}
	$$
	Let $x \in L$. Then $f(x)+g(x) < t$ and more importantly $f(x) < t - g(x)$. Since the rationals are dense in $\RR$ we can say that there exists $r \in \QQ$ such that $f(x) < r < t-g(x)$. Then $f(x) < r$ and $g(x) < t-r$. By definition $x \in R$. \bigbreak
	
	Now let $x \in R$. Then there exists some $r \in \QQ$ such that $x \in f^{-1}\big( (-\infty,r) \big)$ and $x \in g^{-1}\big( (-\infty,t-r) \big)$. Then $f(x)<r$ and $g(x)< t-r$ so $f(x) < r < t-g(x)$, and $f(x)+g(x) < r < t$. Thus $x \in L$. \bigbreak
	
	A similar argument shows that the preimage of an open ray of the form $(t, \infty)$ is $\mu$-measurable. Arbitrary open intervals are intersections of open rays, and since $\BB$ is closed under intersection, the preimage of open intervals is in $\BB$. Every open set can be represented as the countable union of open intervals. Since $\BB$ is closed under countable unions, and countable unions of countable unions are still countable unions (my set theory friends assure me), $f+g$ is $\mu$-measurable. \bigbreak
	
	Now we must show that $f*g$ is $\mu$-measurable. First suppose that $f$ and $g$ are non-negative, and consider the open ray $t,\infty)$. If $t<0$ then this is trivial. Suppose $t>0$. Then it is the case that
	$$
	(fg)^{-1}\big( (t,\infty) \big) = \bigcup_{r \in \QQ} \Big[ f^{-1}\big( (r, \infty) \big) \cap g^{-1}\big( (\tfrac{t}{r}, \infty) \big) \Big]
	$$
	This is shown similarly to the above case, and for the same reasons shows that $fg$ is $\mu$-measurable. Now we consider the open ray $(-\infty, t)$. Since $f$ and $g$ are non-negative, $fg$ is non-negative and 
	$$
	(fg)^{-1}\big( (-\infty, t) \big ) = (fg)^{-1}\big( [0, t) \big ) = (fg)^{-1}\big( \{0\} \big ) \cup (fg)^{-1}\big( (0, t) \big )
	$$
	As above, $(fg)^{-1}\big( \{0\} \big ) = f^{-1}\big( \{0\} \big ) \cup g^{-1}\big( \{0\} \big )$ is $\mu$-measurable. Also similarly to before
	$$
	(fg)^{-1}\big( (0, t) \big ) = \bigcup_{r \in \QQ} \Big[ f^{-1}\big( (0, r) \big) \cap g^{-1}\big( (0,\tfrac{t}{r}) \big) \Big]
	$$
	is $\mu$-measurable. \bigbreak
	
	Finally, for arbitrary $f$ and $g$, 
	$$
	fg = (f^+ - f^-)(g^+ - g^-) = (f^+g^+ + f^-g^-) - (f^+g^- + f^-g^+)
	$$
	This is simply the finite sum of products of non-negative functions and is thus measurable. There is the small issue of the subtraction. That is easily overcome since $n(x)=-x$ is continuous, and hence $f^{-1}(n^{-1}(O))$ is measurable for any open set $O$. 
	
\end{proof}


\bigbreak
%%%%%%%%%%%%%%%%%%%%%%%%%%%%%%%%%%%%%%%%%%%%%%%%%%%%%%%%%%%%%%%%%%%%%%%%%%%%%%%%%%%%%%%%%%%%%%%%%%%%
\problem{16.2} Establish the support truncation property: if $A_n$ is a sequence of measurable sets such that $A_n\subseteq A_{n+1}$ and $\bigcup A_n=X$, then $\int f\chi_{A_n} d\mu\to\int f d\mu$.
\bigbreak

\begin{proof}
	Let $(X,\BB,\mu)$ be a measure space, and let $f:X \to [0,\infty]$ be a $\mu$-measurable function. Let $A_n$ be a sequence of measurable functions such that $A_n \subseteq A_{n+1}$ for all $n \in \NN$ and $\bigcup A_n = X$. Define $f_n = f\chi_{A_n}$. \bigbreak
	
	Let $\epsilon > 0$ be given. By definition, there exists a simple function $g:X \to [0,\infty]$ such that $\int f d\mu - \sint g d\mu < \tfrac{\epsilon}{3}$. \bigbreak
	
	By definition $g = \sum_{i=0}^k c_i \chi_{B_i}$ for some finite number of constants $c_i$ and $\mu$-measurable sets $B_i$. Define $g_n = g\chi_{A_n}$. Then notice that
	$$
	g_n = g\chi_{A_n} = \chi_{A_n} \sum_{i=0}^k c_i \chi_{B_i} =  \sum_{i=0}^k c_i \chi_{A_n} \chi_{B_i} = \sum_{i=0}^k c_i \chi_{B_i \cap A_n}
	$$
	are simple functions for every $n$. Also, by monotonicity
	$$
	\sint g_n d\mu = \sum c_i \mu(B_i \cap A_n) \leq \sum c_i \mu(B_i \cap A_{n+1}) = \sint g_{n+1} d\mu
	$$
	By upward monotone convergence $\sint g_n d\mu \to \sint g d\mu $. Then there exists some $N \in \NN$ such that for any $n \geq N$, $\sint g d\mu - \sint g_n < \tfrac{\epsilon}{3}$. \bigbreak
	
	Lastly note that $(g-f)\chi_{A_n} \leq (g-f)$ so 
	$$
	\sint g_n d\mu - \int f_n d\mu = \int(g-f)\chi_{A_n} d\mu \leq \int (g-f) d\mu \leq \tfrac{\epsilon}{3}
	$$
	Then for all $n \geq N$, consider
	\begin{align*}
		\Bigg \vert \int f d\mu - \int f_n d\mu \Bigg\vert 
		& = \int f d\mu - \int g d\mu + \int g d\mu - \int g_n d\mu + \int g_n d\mu - \int f_n d\mu \\
		& = \int f d\mu - \sint g d\mu + \sint g d\mu - \sint g_n d\mu + \sint g_n d\mu - f_n d\mu \\
		& = \tfrac{\epsilon}{3} + \tfrac{\epsilon}{3} + \tfrac{\epsilon}{3} \\
		& = \epsilon
	\end{align*}
	
\end{proof}


\end{document}
