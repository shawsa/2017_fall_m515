\documentclass[12pt]{article}


% Math		****************************************************************************************
\usepackage{fancyhdr} 
\usepackage{amsfonts}
\usepackage{amsmath}
\usepackage{amssymb}
\usepackage{amsthm}
\usepackage{dsfont}

% Macros	****************************************************************************************
\usepackage{calc}

% Commands and Custom Variables	********************************************************************
\newcommand{\problem}[1]{\hspace{-4 ex} \large \textbf{#1}\\}
\let\oldemptyset\emptyset
\let\emptyset\varnothing
\newcommand{\norm}[1]{\left\lVert#1\right\rVert}
\newcommand{\sint}{\text{s}\kern-5pt\int}

%page		****************************************************************************************
\usepackage[margin=1in]{geometry}
\usepackage{setspace}
\doublespacing
\allowdisplaybreaks
\pagestyle{fancy}
\fancyhf{}
\rhead{Shaw \space \thepage}
\setlength\parindent{0pt}

%Code		****************************************************************************************
\usepackage{listings}
\usepackage{courier}
\lstset{
	language=Python,
	showstringspaces=false,
	formfeed=newpage,
	tabsize=4,
	commentstyle=\itshape,
	basicstyle=\ttfamily,
}

%Images		****************************************************************************************
\usepackage{graphicx}
\graphicspath{ {images/} }

%Hyperlinks	****************************************************************************************
%\usepackage{hyperref}
%\hypersetup{
%	colorlinks=true,
%	linkcolor=blue,
%	filecolor=magenta,      
%	urlcolor=cyan,
%}


\begin{document}
	\thispagestyle{empty}
	
	\begin{flushright}
		Sage Shaw \\
		m515 - Fall 2017 \\
		\today
	\end{flushright}
	
{\large \textbf{HW 5}}\bigbreak

\problem{9.1} Show that a function $f$ is simple if and only if it can be expressed as $f=\sum_1^kc_i\chi_{A_i}$, where $A_i$ are (not necessarily disjoint) Lebesgue measurable sets.

	\begin{proof}
		Suppose that $f=\sum_1^kc_i\chi_{A_i}$ where $A_i$ are measurable sets.
		Let $N = \mathbb{P}(\{1,...,k\})$ (power set). Clearly $N$ is finite. Let $N_i$ for $i=1,...,L$ denote the elements of $N$. Define
		$$
		B_i = \bigcap_{j \in N_i}A_j \setminus \bigcup_{j \notin N_i}A_j
		$$
		and define
		$$
		B_{L+1} = \mathbb{R}^d \setminus \bigcup_{i=1}^k A_i
		$$
		Note that since each $A_j$ is measurable $B_i$ is measurable. It's easy to see that the collection of $B_i$ is pairwise disjoint and $\bigcup_{i=1}^{L+1} B_i = \mathbb{R}^d$. If $x \in B_{L+1}$ then $f(x)=\sum_1^kc_i 0 = 0$ a constant value. If $x \in B_j$ for $j=1,...,L$, then $f(x) = \sum_{i \in N_j}c_i$ which is also a constant value. Since ${B_i}_{i=1}^{L+1}$ partition $\mathbb{R}^d$ into finitely many measurable sets on which $f$ takes a constant value, $f$ is simple by definition. \bigbreak
		
		The other direction is somewhat trivial. Suppose $f$ is simple. Then $f$ it assumes a constant value $c_i$ over a finite number of $A_i$. Then $f = \sum c_i \chi_{A_i}$.
	\end{proof}

\problem{9.2} Show that the simple integral satisfies the properties:
	\begin{itemize}
		\item \emph{finiteness}: $\sint f<\infty$ if and only if $f$ is finite almost everywhere and supported on a set of finite measure
		\item \emph{vanishing}: $\sint f=0$ if and only if $f=0$ almost everywhere
		\item \emph{normality}: $\sint\chi_A=m(A)$ for any Lebesgue measurable set $A$
	\end{itemize}

	\emph{finitiness}: \\
	\begin{proof}
		Suppose that $\sint f < \infty$. Then $f=\sum_1^kc_i m(A_i) < \infty$ implies that $c_i m(A_i) < \infty$ for each $i=1,...,k$. If $c_i$ is infinite then $m(A_i) = 0$. Thus $f$ is finite almost everywhere. If $m(A_i)$ is infinite then $c_i = 0$. Thus $f$ is supported on a set of finite measure (measure of union of $A_i$s is finite by subadditivity).\bigbreak
		
		Suppose that $f$ (a simple function) is finite almost everywhere and supported on a set of finite measure. Then for each $A_i$ on which $f$ assumes a constant value $c_i = \infty$ implies that $m(A_i) = 0$ (finite almost everywhere), and $m(A)=\infty$ implies that $c_i=0$ (supported on a set of finite measure). Thus $c_i m(A_i)$ is finite, and the finite sum of finite numbers is finite. Thus $\sint f < \infty$.
	\end{proof}

	\emph{vanishing}: \\
	\begin{proof}
		Suppose that $\sint f=0$. Since $f$ is nonnegative, $\sint f=0$ implies that each $c_i m(A_i)$ is zero for $i=1,...,k$. Thus $c_i = 0$ or $m(A_i)=0$. If $c_i \neq 0$ then $m(A_i)=0$. Since the union of Lebegue null sets is Lebesgue null (subaddititivity) $f=0$ almost everywhere. \bigbreak
		Suppose that $f=0$ almost everywhere. Then for $\sint f = \sum_1^k c_i m(A_i)$, we know that if $c_i \neq 0$ that $m(A_i) = 0$. Thus $c_i m(A_i) = 0$ for each $i=1,...,k$ and the sum of zeros is zero. Thus $\sint f = 0$.
	\end{proof}

	\emph{normality}: \\
	\begin{proof}
		Let $A$ be a Lebesgue measurable set. Then $A^c$ is Lebesgue measurable as well and $\{A, A^c\}$ partition $\mathbb{R}^d$. Then $$\sint \chi_A = 1*m(A) + 0*m(A^c) = m(A)$$
	\end{proof}
	

\problem{10.1} Show that $\lim x_n=x$ if and only if $\liminf x_n=x=\limsup x_n$.

	\begin{proof}
		Let $\epsilon >0$ be given and suppose that $\liminf{x_n} = x = \limsup{x_n}$. Since $\liminf{x_n} = x = \sup\limits_N \inf\limits_{n \geq N}x_n$, we know there exists $N_1$ such that \\$x - \inf\limits_{n \geq N_1} < \epsilon$. Then $x - \epsilon < \inf\limits_{n \geq N_1} \leq x_n$ and $x - x_n < \epsilon$ for $N_1 \leq n$. Since $\limsup{x_n} = x = \inf\limits_N \sup\limits_{n \geq N}x_n$, we know there exists $N_2$ such that \\$\sup\limits_{n \geq N_2} - x < \epsilon$. Then $x_n \leq \sup\limits_{n \geq N_2}x_n < x + \epsilon$ and $-\epsilon < x - x_n$ for $N_2 \leq n$. Let $N = \max\{N_1,N_2\}$. Then $-\epsilon < x - x_n < \epsilon$ and $\vert x - x_n \vert < \epsilon$. Thus $\lim x_n = x$. \bigbreak
		
		Suppose that $\lim x_n = x$ and $\liminf x_n = y$. Then for all $\epsilon > 0$ there exisits $N_1$ such that $y -  \inf\limits_{n \geq N_1}x_n < \tfrac{\epsilon}{2}$. Since $\lim x_n = x$ there exists some $N_2$ such that $\vert x - x_n \vert < \tfrac{\epsilon}{2}$ for all $n \geq N_2$. Let $N = \max\{N_1,N_2\}$. Then for all $n \geq N$
		\begin{align*}
			\vert y - x \vert & = \vert y - \inf\limits_{n \geq N}x_n + \inf\limits_{n \geq N}x_n - x \vert \\
			& \leq \vert y - \inf\limits_{n \geq N}x_n \vert + \vert \inf\limits_{n \geq N}x_n - x \vert \\
			& \leq \vert y - \inf\limits_{n \geq N_1}x_n \vert + \vert \inf\limits_{n \geq N}x_n - x \vert \\
			& < \tfrac{\epsilon}{2} + \tfrac{\epsilon}{2} = \epsilon
		\end{align*}
		Thus $\liminf x_n = y = x$. \bigbreak
		Suppose now that $\limsup x_n = w$. Then for all $\epsilon > 0$ there exisits $N_1$ such that $\sup\limits_{n \geq N_1}x_n - w < \tfrac{\epsilon}{2}$. Since $\lim x_n = x$ there exists some $N_2$ such that $\vert x - x_n \vert < \tfrac{\epsilon}{2}$ for all $n \geq N_2$. Let $N = \max\{N_1,N_2\}$. Then for all $n \geq N$
		\begin{align*}
			\vert x - w \vert & = \vert x - \sup\limits_{n \geq N}x_n + \sup\limits_{n \geq N}x_n - w \vert \\
			& \leq \vert x - \sup\limits_{n \geq N}x_n \vert + \vert \sup\limits_{n \geq N}x_n - w \vert \\
			& \leq \vert x - \sup\limits_{n \geq N}x_n \vert + \vert \sup\limits_{n \geq N_1}x_n - w \vert \\
			& < \tfrac{\epsilon}{2} + \tfrac{\epsilon}{2} = \epsilon
		\end{align*}
		Thus $\limsup x_n = w = x$. \bigbreak
	\end{proof}

\problem{10.3} Show that if $f$ is a bounded, nonnegative measurable function on $\mathbb{R}^d$, then there is a sequence of bounded simple functions $f_n$ which converges uniformly to $f$ (not just pointwise).

	\begin{proof}
		Since $f$ is bounded and nonnegative, there exists $B$ such that \\$f(\mathbb{R}^d) \subseteq [0,B)$. Define $A_{n,i} = f^{-1}\Big( \big[\tfrac{Bi}{n}, \tfrac{B(i+1)}{n}\big)\Big)$ for $n \in \mathbb{N}$ and for $i=0,...,n-1$. Note that each $A_{n,i}$ is measurable since $\big[\tfrac{Bi}{n}, \tfrac{B(i+1)}{n}\big)$ can be written as a countable intersection of open sets. Also note that $\{A_{n,i}\}_{i=0}^{n-1}$ is a partition of the range of $f$ for each $n$. Now define 
		$$f_n(x) = \Big\{\tfrac{Bi}{n} \text{ if }x\in A_n,i$$
		Let $\epsilon > 0$ be given. By the Archemidian Property there exists $N \in \mathbb{N}$ such that $\tfrac{1}{\epsilon} < N$. Then for all $n\in \mathbb{N}$ such that $N \leq n$ and for all $x \in mathbb{R}^d$ there exists $i \in {0,...,n-1}$ such that $x \in A_{n,i}$. By definition $f(x) \in \big[\tfrac{Bi}{n}, \tfrac{B(i+1)}{n}\big)$ and $f_n(x) = \tfrac{Bi}{n}$. Clearly
		$\vert f(x) - f_n(x) \vert < \tfrac{1}{n} \leq \tfrac{1}{N} < \epsilon$.
	\end{proof}

\problem{10.4} Let $f$ be a nonnegative function on $\mathbb{R}^d$. Show that $f$ is simple if and only if $f$ is measurable and takes on at most finitely many values.

	\begin{proof}
		If $f$ is simple clearly it takes on finitely many values and $\lim f = f$ thus $f$ is measurable. \bigbreak
		
		Suppose that $f$ is measurable and assumes finitely many values. 
		Then $f(\mathbb{R}^d) = \{c_i\}_{i-1}^k$ for some some finite collection of distinct $c_i$. 
		Since $f$ is measurable $f^{-1}\big(\{c_i\}\big)$ is Lebesgue measurable (since each point can be written as a countable union of open sets). 
		Since $f$ is a well-defined function and our $c_i$ are distinct $f^{-1}\big(\{c_i\}\big) \cap f^{-1}\big(\{c_j\}\big) = \emptyset $ if $i \neq j$.
		Clearly $\Big \{f \big(\{c_i\} \big) \Big\}_{i=1}^k$ is a partition of $\mathbb{R}^d$. Thus $f$ is simple. 
	\end{proof}


\end{document}
