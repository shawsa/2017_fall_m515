\documentclass[12pt]{article}

\usepackage{fancyhdr} 
%\usepackage{mathtools}
\usepackage{amsmath}
\usepackage{amsthm}
\usepackage{dsfont}

\pagestyle{fancy}
\fancyhf{}
\rhead{Shaw \space \thepage}

\setlength\parindent{0pt}

\begin{document}
	\thispagestyle{empty}
	
	\begin{flushright}
		Sage Shaw \\
		m527 - Fall 2017 \\
		Aug. 30, 2017
	\end{flushright}
	
	{\large \textbf{HW1: 1.1-3, 2.1-2}}\bigbreak

	\hspace{-4 ex}\textbf{1.1} Show that the properties (a)--(c) of a measure imply finite addivitity: If $A$ and $B$ are disjoint then $m(A\cup B)=m(A)+m(B)$.
	
	%\bigbreak
	
	\begin{proof}
		 Let $A$ and $B$ be disjoint sets. Define $A_{1}=A$, $A_{2}=B$ and $A_{n}=\emptyset$ for $n>2$.
	
	Clearly all $A_{n}$ are pairwise disjoint.
	
	Furthermore $A \cup B = \bigcup_{n=1}^{\infty}A_{n}$, and $B = \cup_{n=2}^{\infty}A_{n}$.
	
	Then
	\begin{equation*}
	\begin{split}
	m(A \cup B) & = m(\bigcup\limits_{n=1}^{\infty}A_{n}) \\
	& = \sum_{n=1}^{\infty}m(A_{n}) \text{ by property (c)} \\
	& = m(A_{1}) + \sum_{n=2}^{\infty}m(A_{n} \\
	& = m(A) + m(\bigcup\limits_{n=2}^{\infty}A_{n}) \text{ by property (c)} \\
	& = m(A) + m(B)
	\end{split}
	\end{equation*}
	
	\end{proof}

\bigbreak
\bigbreak



	\hspace{-4 ex}\textbf{1.2} Show that the properties (a)--(c) of a measure imply the inclusion--exclusion principle: For any sets $A,B$ we have $m(A\cup B)+m(A\cap B)=m(A)+m(B)$.
	
	Note the following equations from set theory:
	
	\begin{equation}
	A = (A \setminus B) \cup (A \cap B)
	\end{equation}
	\begin{equation}
	B = (B \setminus A) \cup (A \cap B)
	\end{equation}
	\begin{equation}
	A \cup B = (A \setminus B) \cup (B \setminus A) \cup (A \cap B)
	\end{equation}
	
	\begin{proof}
		\begin{align*}
		m(A) + m(B) & = m\Big((A \setminus B) \cup (A \cap B)\Big) + m\Big((B \setminus A) \cup (A \cap B)\Big) \\
		& \hspace{20 ex} \text{by (1) and (2)} \\
		& = m(A \setminus B) + m(A \cap B) + m(B \setminus A) + m(A \cap B) \\
		& \hspace{20 ex} \text{by finite additivity} \\
		& = \Big( m(A \setminus B) + m(A \cap B) + m(B \setminus A) \Big) +  m(A \cap B) \\
		& = m\Big( (A \setminus B) \cup (A \cap B) \cup (B \setminus A) \Big) +  m(A \cap B)\\
		& \hspace{20 ex}  \text{by finite additivity} \\
		& = m(A \cup B) + m(A \cap B) \\
		& \hspace{20 ex} \text{by (3)}
		\end{align*}
	\end{proof}

\bigbreak
\bigbreak

	\hspace{-4 ex}\textbf{2.1}Show that the class of elementary sets is closed under the operations: union, intersection, set difference, symmetric difference, and translation. \bigbreak
	
	Given elementary sets $A$ and $B$: \bigbreak
	
	(a) $A \cup B$ is an elementary set.
	
	
	\begin{proof}
		Since $A$ is an elementary set, $A = \bigcup_{n=1}^{N} A{n}$ where each $A_{n}$ is a box for $n=1,...,N$. \\
		Since $B$ is an elementary set, $B = \bigcup_{n=1}^{M} B{n}$ where each $B_{n}$ is a box for $n=1,...,M$. \\
		Define $C_{n} = A_{n}$ for $n=1,...,N$ and $C_{n} = B_{n-N}$ for $n=N+1,...N+M$. \\
		Clearly $A \cup B = \bigcup_{n=1}^{N+M} C{n}$ where each $C_{n}$ is a box. \\
		Therefore $A \cup B$ is an elementary set.
	\end{proof}

	(b) $A \cap B$ is an elementary set. \bigbreak
	
	\textbf{Lemma:} the intersection of two intervals is an interval.
	
	\begin{proof}
		Let $[a,b]$ and $[c,d]$ be intervals. \bigbreak
		Suppose $a \leq c$. \\
		If $ b < c$ then clearly $[a,b] \cap [c,d] = \emptyset$ which is an interval. \\
		If $b=c$ then $[a,b] \cap [c,d] = [b,b]$. \\
		Now further suppose $ c < b$. \\
		Then if $b \leq d$, clearly $[a,b] \cap [c,d] = [c,b]$. \\
		Otherwise, $ d < b$ and $[a,b] \cap [c,d] = [c,d]$. \bigbreak
		The cases when $c \leq a$ are similar to the above cases. \bigbreak
		
		The intersection of any two kinds of intervals (15 other) will be similar to the above case of the intersection of two closed intvervals, but with different inequalities. \bigbreak
		
		Hence, the intersection of two intervals is an interval.		
	\end{proof}

	\textbf{Lemma:} the intersection of two boxes is a box.
	
	\begin{proof}
		Let $A = I_{1} \times ... \times I_{d}$ and $B = J_{1} \times ... \times J_{d}$ be boxes in $\mathds{R}^{d}$. \\
		It follows that $A \cap B = (I_{1} \cap J_{1}) \times ... \times (I_{d} \cap J_{d})$. \\
		From the above lemma, $(I_{n} \cap J_{n})$ are intervals for $n=1,...,d4$.
	\end{proof}

	We can now easily prove that the intersection of two elementary sets is an elementary set.
	
	\begin{proof}
		Let $A$ and $B$ be elementary sets. \\
		Then $A=\bigcup\limits_{n=1}^{N}A_{n}$ and $b=\bigcup\limits_{m=1}^{M}B_{m}$ where each $A_{n}$ and $B_{m}$ are boxes. \\
		Let $x \in A \cap B$.
		Then there exists $n \in \{1,...,N\}$ and $m \in \{1,...,M\}$ such that $x \in A_{n}$ and $x \in B_{m}$. \\
		Therefore $x \in A_{n} \cap B_{m}$, and $x \in \bigcup\limits_{(n,m)=(1,1)}^{(N,M)}(A_{n} \cap B_{m})$. \bigbreak
		
		Let $x \in \bigcup\limits_{(n,m)=(1,1)}^{(N,M)}(A_{n} \cap B_{m})$. \\
		Then $x \in A_{n}$ and $x \in B_{n}$ for some $n \in \{1,...,N\}$ and $m \in \{1,...,M\}$. \\
		Thus $x \in A \cap B$. \bigbreak
		Since  $A \cap B = \bigcup\limits_{(n,m)=(1,1)}^{(N,M)}(A_{n} \cap B_{m})$, we can say that the intersection of two elementary sets is an elementary set, by the above lemma.
	\end{proof}

\end{document}
