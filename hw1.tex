\documentclass[12pt]{article}

\usepackage{fancyhdr} 
\usepackage{amsfonts}
\usepackage{amsmath}
\usepackage{amsthm}
\usepackage{dsfont}

\pagestyle{fancy}
\fancyhf{}
\rhead{Shaw \space \thepage}

\setlength\parindent{0pt}

\begin{document}
	\thispagestyle{empty}
	
	\begin{flushright}
		Sage Shaw \\
		m515 - Fall 2017 \\
		Sep. 1, 2017
	\end{flushright}
	
	{\large \textbf{HW 1: 1.1-3, 2.1-2}}\bigbreak

	\hspace{-4 ex}\textbf{1.1} Show that the properties (a)--(c) of a measure imply finite addivitity: If $A$ and $B$ are disjoint then $m(A\cup B)=m(A)+m(B)$.
	
	%\bigbreak
	
	\begin{proof}
		 Let $A$ and $B$ be disjoint sets. Define $A_{1}=A$, $A_{2}=B$ and $A_{n}=\emptyset$ for $n>2$.
	
	Clearly all $A_{n}$ are pairwise disjoint.
	
	Furthermore $A \cup B = \bigcup_{n=1}^{\infty}A_{n}$, and $B = \cup_{n=2}^{\infty}A_{n}$.
	
	Then
	\begin{equation*}
	\begin{split}
	m(A \cup B) & = m(\bigcup\limits_{n=1}^{\infty}A_{n}) \\
	& = \sum_{n=1}^{\infty}m(A_{n}) \text{ by property (c)} \\
	& = m(A_{1}) + \sum_{n=2}^{\infty}m(A_{n} \\
	& = m(A) + m(\bigcup\limits_{n=2}^{\infty}A_{n}) \text{ by property (c)} \\
	& = m(A) + m(B)
	\end{split}
	\end{equation*}
	
	\end{proof}

\bigbreak
\bigbreak



	\hspace{-4 ex}\textbf{1.2} Show that the properties (a)--(c) of a measure imply the inclusion--exclusion principle: For any sets $A,B$ we have $m(A\cup B)+m(A\cap B)=m(A)+m(B)$.
	
	Note the following equations from set theory:
	
	\begin{equation}
	A = (A \setminus B) \cup (A \cap B)
	\end{equation}
	\begin{equation}
	B = (B \setminus A) \cup (A \cap B)
	\end{equation}
	\begin{equation}
	A \cup B = (A \setminus B) \cup (B \setminus A) \cup (A \cap B)
	\end{equation}
	
	\begin{proof}
		\begin{align*}
		m(A) + m(B) & = m\Big((A \setminus B) \cup (A \cap B)\Big) + m\Big((B \setminus A) \cup (A \cap B)\Big) \\
		& \hspace{20 ex} \text{by (1) and (2)} \\
		& = m(A \setminus B) + m(A \cap B) + m(B \setminus A) + m(A \cap B) \\
		& \hspace{20 ex} \text{by finite additivity} \\
		& = \Big( m(A \setminus B) + m(A \cap B) + m(B \setminus A) \Big) +  m(A \cap B) \\
		& = m\Big( (A \setminus B) \cup (A \cap B) \cup (B \setminus A) \Big) +  m(A \cap B)\\
		& \hspace{20 ex}  \text{by finite additivity} \\
		& = m(A \cup B) + m(A \cap B) \\
		& \hspace{20 ex} \text{by (3)}
		\end{align*}
	\end{proof}

\bigbreak

\hspace{-4 ex}\textbf{1.3} Show that if there is a metric on $\mathbb{R}^{d}$ that satisfies properties (a)-(c), that there is a metric on $[0,1)$ mod $1$ that does as well. \bigbreak

For $x,y \in [0,1)$ define 
\[ x \oplus y =
	\begin{cases}
	x + y & \text{if } x+y < 1 \\
	x + y -1 & \text{if } x+y \geq 1\\
	\end{cases}
\]

Define $m_{1}:[0,1) \rightarrow \mathds{R}$ as $m_{1}(A) = m(A)$ where $m$ is a measure on $\mathbb{R}$ satisfying (a)-(c). \\

\hspace{-4 ex}\textbf{(a)} Normality

\begin{proof}
	Let $I \in [0,1)$ be an interval. \\
	Then $m_{1}(I) = m(I)$ and thus the length of the measure of the interval is it's length.\\
\end{proof}

\hspace{-4 ex}\textbf{(b)} Transitivity

\begin{proof}
	Let $x \in [0,1)$ and $ A \subseteq [0,1)$. \\
	Define $A_{1}= A \setminus [1-x,1)$ and $A_{2} = A \setminus [0, 1-x)$. \\
	Clearly $A = A_{1} \cup A_{2}$ and $A_{1} \cap A_{2} = \emptyset$. \\
	Then 
	\begin{align*}
		m_{1}(x \oplus A) & = m(x \oplus A) \\
		& = m(x \oplus A_{1} \cup x \oplus A_{2}) \\
		& = m(x \oplus A_{1}) + m(x \oplus A_{2}) \\
		& = m(x + A_{1}) + m( (x-1) + A_{2}) & \text{since } x \oplus A_{2} \subseteq [1,2) \\
		& = m(A_{1}) + m(A_{2}) \\
		& = m(A_{1} \cup A_{2}) \\
		& = m(A) \\
		& = m_{1}(A) \\
	\end{align*} 
\end{proof}

\hspace{-4 ex}\textbf{(c)} Countable Additivity \\

\begin{proof}
	Let $A_{n} \subseteq [0,1)$ for $n \in \mathbb{N}$, such that all $A_{n}$ are pairwise disjoint. \\
	Then $m_{1}\big(\bigcup\limits_{n=1}^{\infty}A_{n}\big) = m \big(\bigcup\limits_{n=1}^{\infty}A_{n}\big) = \sum\limits_{n=1}^{\infty}m(A_{n}) = \sum\limits_{n=1}^{\infty}m_{1}(A_{n})$.\\
\end{proof}


\bigbreak

	\hspace{-4 ex}\textbf{2.1} Show that the class of elementary sets is closed under the operations: union, intersection, set difference, symmetric difference, and translation. \bigbreak
	
	Given elementary sets $A$ and $B$: \bigbreak
	
	\hspace{-4 ex}\textbf{(a)} $A \cup B$ is an elementary set.
	
	
	\begin{proof}
		Since $A$ is an elementary set, $A = \bigcup_{n=1}^{N} A{n}$ where each $A_{n}$ is a box for $n=1,...,N$. \\
		Since $B$ is an elementary set, $B = \bigcup_{n=1}^{M} B{n}$ where each $B_{n}$ is a box for $n=1,...,M$. \\
		Define $C_{n} = A_{n}$ for $n=1,...,N$ and $C_{n} = B_{n-N}$ for $n=N+1,...N+M$. \\
		Clearly $A \cup B = \bigcup_{n=1}^{N+M} C{n}$ where each $C_{n}$ is a box. \\
		Therefore $A \cup B$ is an elementary set.
	\end{proof}

	\hspace{-4 ex}\textbf{(b)} $A \cap B$ is an elementary set. \bigbreak
	
	\textbf{Lemma:} the intersection of two intervals is an interval.
	
	\begin{proof}
		Let $[a,b]$ and $[c,d]$ be intervals. \bigbreak
		Suppose $a \leq c$. \\
		If $ b < c$ then clearly $[a,b] \cap [c,d] = \emptyset$ which is an interval. \\
		If $b=c$ then $[a,b] \cap [c,d] = [b,b]$. \\
		Now further suppose $ c < b$. \\
		Then if $b \leq d$, clearly $[a,b] \cap [c,d] = [c,b]$. \\
		Otherwise, $ d < b$ and $[a,b] \cap [c,d] = [c,d]$. \bigbreak
		The cases when $c \leq a$ are similar to the above cases. \bigbreak
		
		The intersection of any two kinds of intervals (15 other cases) will be similar to the above case of the intersection of two closed intvervals, but with different inequalities. \bigbreak
		
		Hence, the intersection of two intervals is an interval.		
	\end{proof}

	\textbf{Lemma:} the intersection of two boxes is a box.
	
	\begin{proof}
		Let $A = I_{1} \times ... \times I_{d}$ and $B = J_{1} \times ... \times J_{d}$ be boxes in $\mathds{R}^{d}$. \\
		It follows that $A \cap B = (I_{1} \cap J_{1}) \times ... \times (I_{d} \cap J_{d})$. \\
		From the above lemma, $(I_{n} \cap J_{n})$ are intervals for $n=1,...,d4$.
	\end{proof}

	We can now easily prove that the intersection of two elementary sets is an elementary set.
	
	\begin{proof}
		Let $A$ and $B$ be elementary sets. \\
		Then $A=\bigcup\limits_{n=1}^{N}A_{n}$ and $b=\bigcup\limits_{m=1}^{M}B_{m}$ where each $A_{n}$ and $B_{m}$ are boxes. \\
		Let $x \in A \cap B$.
		Then there exists $n \in \{1,...,N\}$ and $m \in \{1,...,M\}$ such that $x \in A_{n}$ and $x \in B_{m}$. \\
		Therefore $x \in A_{n} \cap B_{m}$, and $x \in \bigcup\limits_{(n,m) \in Z}(A_{n} \cap B_{m})$ \\
		Where $Z = \{1,...,N\} \times \{1,...,M\}$. \bigbreak
		
		Let $x \in \bigcup\limits_{(n,m) \in Z}(A_{n} \cap B_{m})$. \\
		Then $x \in A_{n}$ and $x \in B_{m}$ for some $n \in \{1,...,N\}$ and $m \in \{1,...,M\}$. \\
		Thus $x \in A \cap B$. \bigbreak
		Since  $A \cap B = \bigcup\limits_{(n,m) \in Z}(A_{n} \cap B_{m})$, we can say that the intersection of two elementary sets is an elementary set, by the above lemma. \\
	\end{proof}

	\hspace{-4 ex}\textbf{(c)} $A \setminus B$ is an elementary set. \bigbreak
	
	\textbf{Lemma:} The set difference of two intervals is a finite union of intervals.
	
	\begin{proof}
		Let $[a,b]$ and $[c,d]$ be intervals. \bigbreak
		Suppose $a \leq c$. \\
		If $ b < c$ then clearly $[a,b] \setminus [c,d] = [a,b]$ which is an interval. \\
		If $b=c$ then $[a,b] \setminus [c,d] = [a,b)$. \\
		Now further suppose $ c < b$. \\
		Then if $b \leq d$, clearly $[a,b] \setminus [c,d] = [a,c)$. \\
		Otherwise, $ d < b$ and $[a,b] \setminus [c,d] = [a,c) \cup (d,b]$. \bigbreak
		The cases when $c \leq a$ are similar to the above cases. \bigbreak
		
		The set difference of any two kinds of intervals (15 other cases) will be similar to the above case of the set difference of two closed intvervals, but with different inequalities. \bigbreak
		
		Hence, the set difference of two intervals is a finite union of intervals. \\
	\end{proof}

\textbf{Lemma:} The set difference of two boxes is a union of boxes.

\begin{proof}
	Let $A = I_{1} \times ... \times I_{d}$ and $B = J_{1} \times ... \times J_{d}$ be boxes in $\mathds{R}^{d}$. \\
	Define 
	\[ 
		K_{(i,t)} =
		\begin{cases} 
			I_{i} \setminus J_{i} & \text{if } t=1 \\
			I_{i} & \text{if } t=2 \\
		\end{cases}
	\]
	for $i \in \{1,...,d\}$ and $t \in {1,2}$.\\
	Each $K_{(i,t)}$ is a finite union of intervals by a previous lemma.\\
	Define $\vec{v}$ as a $d$ dimensional vector such that $\vec{v}_{i} \in \{1,2\}$. ($\vec{v}$ is a $d$ dimensional vector of $1$s and $2$s.)
	Define $P_{\vec{v}} = \prod\limits_{i=1}^{d} K_{(i,\vec{v}_{i})}$. \\
	Since $(A \cup B) \times C = (A \times C) \cup (B \times C)$ from set theory, we can say that each $P_{\vec{v}}$ is a union of boxes. (A product of unions of intervals is a union of products of intervals). \bigbreak
	
	Let $P = \bigcup\limits_{\vec{u} \in V} P_{\vec{u}}$ where $V$ is the set of $d$ dimensional vectors of $1$s and $2$s except the vector of all $1$s. \\
	Since $P$ is a finite union of finite unions of boxes, it is an elementary set. It remains to show that $A \setminus B = P$. \bigbreak
	
	Let $\vec{x} \in A \setminus B$. \\
	Since $\vec{x} \notin B$, there exists $i \in \{1,...,d\}$ such that $\vec{x}_{i} \notin J_{i}$. \\
	Thus $\vec{x}_{i} \in K_{i,1}$ \\
	For $j \in \{1,...,d\}$ where $j \neq i$ it is clear that $\vec{x}_{j}$ is either in $ K_{j,1}$ or $ K_{j,2}$. \\
	Thus $x \in P_{\vec{v}}$ for some $\vec{v}$ that is not all $2$s, and thus $x \in P$. \bigbreak 
	
	Let $\vec{x} \in P$. \\
	Then $\vec{x} \in P_{\vec{v}}$ for some $\vec{v}$ that is not all $1$s. \\
	Then $\vec{x}_{i} \in K_{i,t}$ for all $i \in \{1,...,d\}$ and $t \in \{1,2\}$. \\
	Clearly $\vec{x} \in A$. \\
	Since $\vec{v}$ is not all $2$s, $\exists j \in \{1,...,d\}$ such that $\vec{x}_{j} \in K_{j,1}$, and thus $\vec{x}_{j} \notin J_{j}$. \\
	Finally $\vec{x} \notin B$, so $\vec{x} \in A \setminus B$.	\\
\end{proof}

	Now we are able to prove that if $A$ and $B$ are elementary sets, then $A \setminus B$ is an elementary set. \\
	
	\begin{proof}
		Let $A$ and $B$ be elementary sets. \\
		Then $A = \bigcup\limits_{n=1}^{N} A_{n}$ where each $A_{n}$ is elementary. \\
		Similarly $B = \bigcup\limits_{m=1}^{M} B_{m}$ where each $B_{m}$ is elementary. \\
		It follows that $A \setminus B = \bigcup\limits_{n=1}^{N} A_{n} \setminus \bigcup\limits_{m=1}^{M} B_{m} = \bigcup\limits_{n=1}^{N}  \bigcup\limits_{m=1}^{M} A_{n} \setminus B_{m}$. \\
		Since this is a finite unioin of elementary sets, $A \setminus B$ is an elementary set. \\
	\end{proof}

\hspace{-4 ex}\textbf{(d)} $A \Delta B = (A \setminus B) \cup (B \setminus A)$ is an elementary set.
\begin{proof}
	Let $A$ and $B$ be elementary sets. \\
	By the previous proof, $A \setminus B$ and $B \setminus A$ are also elementary sets. \\
	By part (a), $(A \setminus B) \cup (B \setminus A)$ is an elementary set.\\
\end{proof}

\hspace{-4 ex}\textbf{(e)} For $\vec{x} \in \mathbb{R}^{d}$, $\vec{x} + A$ is an elementary set. \bigbreak

\begin{proof}
	For any interval $(a,b)$, and any $x \in \mathbb{R}$, clearly $x + (a,b) = (x+a, x+b)$ which is still an interval. This holds for all four types of intervals. \bigbreak
	
	For any box $B = \prod\limits_{i=1}^{d}I_{i}$, and any $\vec{x} \in \mathbb{R}^{d}$, clearly $x + B = \prod\limits_{i=1}^{d}x_{i}+I_{i}$ which is still a box. \bigbreak
	
	For any elementary set $A = \bigcup\limits_{i=1}^{n}B_{i}$, and any $\vec{x} \in \mathbb{R}^{d}$, clearly \\
	\begin{align}
		x + A & = x + \bigcup\limits_{i=1}^{d}B_{i} \\
		& = \bigcup\limits_{i=1}^{d}x + B_{i} \\
	\end{align}
	and is still a finite union of boxes. \\
 \bigbreak
	
	
\end{proof}


\hspace{-4 ex}\textbf{2.2}Prove the elementary measure satisfies the monotonicity and finite subadditivity properties. \bigbreak

Let $A$ and $B$ be elementary sets. \bigbreak

\hspace{-4 ex}\textbf{(a)} Show that if $A \subseteq B$ then $m(A) \leq m(B)$. \bigbreak

\begin{proof}
	Note that $(B \setminus A) \cup A = B$, since $A \subseteq B$. \\
	Then $m(B \setminus A) + m(A) = m(B)$, since $A$ and $B \setminus A$ are disjoint. \\
	Since $m(B\setminus A) \geq 0$ by definition of measure and $m(A) \leq m(B)$.\\	
\end{proof}

\hspace{-4 ex}\textbf{(b)} Show that $m(A \cup B) \leq m(A) + m(B)$. \bigbreak

\begin{proof}
	Exercise 1.2 states that for any sets $A$ and $B$, \\ $m(A\cup B)+m(A\cap B)=m(A)+m(B)$. \\
	Then $m(A\cap B) = m(A) + m(B) - m(A\cup B)$ \\
	By definition of measure, $0 \leq m(A \cap B)$. \\
	So $0 \leq m(A) + m(B) - m(A\cup B)$ and finally, $m(A\cup B) \leq m(A)+m(B)$. \\
\end{proof}

\end{document}
