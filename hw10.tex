\documentclass[12pt]{article}

%%%%%%%%%%%%%%%%%%%%%%%%%%%%%%%%%%%%%%%%%%%%%%%%%%%%%%%%%%%%%%%%%%%%%%%%%%%%%%%%%%%%%%%%%%%%%%%%%%%%
% Math
\usepackage{fancyhdr} 
\usepackage{amsfonts}
\usepackage{amsmath}
\usepackage{amssymb}
\usepackage{amsthm}
%\usepackage{dsfont}

%%%%%%%%%%%%%%%%%%%%%%%%%%%%%%%%%%%%%%%%%%%%%%%%%%%%%%%%%%%%%%%%%%%%%%%%%%%%%%%%%%%%%%%%%%%%%%%%%%%%
% Macros
\usepackage{calc}

%%%%%%%%%%%%%%%%%%%%%%%%%%%%%%%%%%%%%%%%%%%%%%%%%%%%%%%%%%%%%%%%%%%%%%%%%%%%%%%%%%%%%%%%%%%%%%%%%%%%
% Commands and Custom Variables	
\newcommand{\problem}[1]{\hspace{-4 ex} \large \textbf{Problem #1} }
\let\oldemptyset\emptyset
\let\emptyset\varnothing
\newcommand{\norm}[1]{\left\lVert#1\right\rVert}
\newcommand{\sint}{\text{s}\kern-5pt\int}
\newcommand{\powerset}{\mathcal{P}}
\renewenvironment{proof}{\hspace{-4 ex} \emph{Proof}:}{\qed}
\newcommand{\RR}{\mathbb{R}}
\newcommand{\NN}{\mathbb{N}}
\newcommand{\QQ}{\mathbb{Q}}
\newcommand{\ZZ}{\mathbb{Z}}
\newcommand{\CC}{\mathbb{C}}


%%%%%%%%%%%%%%%%%%%%%%%%%%%%%%%%%%%%%%%%%%%%%%%%%%%%%%%%%%%%%%%%%%%%%%%%%%%%%%%%%%%%%%%%%%%%%%%%%%%%
%page
\usepackage[margin=1in]{geometry}
\usepackage{setspace}
%\doublespacing
\allowdisplaybreaks
\pagestyle{fancy}
\fancyhf{}
\rhead{Shaw \space \thepage}
\setlength\parindent{0pt}

%%%%%%%%%%%%%%%%%%%%%%%%%%%%%%%%%%%%%%%%%%%%%%%%%%%%%%%%%%%%%%%%%%%%%%%%%%%%%%%%%%%%%%%%%%%%%%%%%%%%
%Code
\usepackage{listings}
\usepackage{courier}
\lstset{
	language=Python,
	showstringspaces=false,
	formfeed=newpage,
	tabsize=4,
	commentstyle=\itshape,
	basicstyle=\ttfamily,
}

%%%%%%%%%%%%%%%%%%%%%%%%%%%%%%%%%%%%%%%%%%%%%%%%%%%%%%%%%%%%%%%%%%%%%%%%%%%%%%%%%%%%%%%%%%%%%%%%%%%%
%Images
\usepackage{graphicx}
\graphicspath{ {images/} }
\usepackage{float}

%tikz
\usepackage[utf8]{inputenc}
\usepackage{pgfplots}
\usepgfplotslibrary{groupplots}

%%%%%%%%%%%%%%%%%%%%%%%%%%%%%%%%%%%%%%%%%%%%%%%%%%%%%%%%%%%%%%%%%%%%%%%%%%%%%%%%%%%%%%%%%%%%%%%%%%%%
%Hyperlinks
%\usepackage{hyperref}
%\hypersetup{
%	colorlinks=true,
%	linkcolor=blue,
%	filecolor=magenta,      
%	urlcolor=cyan,
%}

\begin{document}
	\thispagestyle{empty}
	
	\begin{flushright}
		Sage Shaw \\
		m515 - Fall 2017 \\
		\today
	\end{flushright}
	
{\large \textbf{HW 10}}\bigbreak

%%%%%%%%%%%%%%%%%%%%%%%%%%%%%%%%%%%%%%%%%%%%%%%%%%%%%%%%%%%%%%%%%%%%%%%%%%%%%%%%%%%%%%%%%%%%%%%%%%%%
\problem{Ex 19.1.} Show that $\norm{T}$ is equal to $\sup\{\norm{Tx}:\norm{x}\leq1\}$. \bigbreak


\bigbreak
%%%%%%%%%%%%%%%%%%%%%%%%%%%%%%%%%%%%%%%%%%%%%%%%%%%%%%%%%%%%%%%%%%%%%%%%%%%%%%%%%%%%%%%%%%%%%%%%%%%%
\problem{Ex 19.3 (BBT, ex 17:7.5).} If $X,Y$ are Banach spaces and $T\in B(X,Y)$, show that $(T^*\phi)(x)=\phi(Tx)$ defines an element of $B(Y^*,X^*)$ such that $\norm{T^*}=\norm{T}$. \bigbreak

\bigbreak
%%%%%%%%%%%%%%%%%%%%%%%%%%%%%%%%%%%%%%%%%%%%%%%%%%%%%%%%%%%%%%%%%%%%%%%%%%%%%%%%%%%%%%%%%%%%%%%%%%%%
\problem{Ex 19.5.} Complete the proof of Corollary 19.4(b) of the notes: If $Y$ is closed in $X$ and $z\notin Y$, then there exists $\phi\in X^*$ such that $\phi(Y)=0$, $\phi(z)\neq0$, and $\norm{\phi}=1$. \bigbreak

\bigbreak
%%%%%%%%%%%%%%%%%%%%%%%%%%%%%%%%%%%%%%%%%%%%%%%%%%%%%%%%%%%%%%%%%%%%%%%%%%%%%%%%%%%%%%%%%%%%%%%%%%%%
\problem{Ex 20.1.} Give an example of a function from $\RR$ to $\RR$ which has a closed graph but is not continuous. Give an example of function from $\RR$ to $\RR$ which is continuous and surjective but not open. Is it possible to give a bijective example? \bigbreak


\bigbreak
%%%%%%%%%%%%%%%%%%%%%%%%%%%%%%%%%%%%%%%%%%%%%%%%%%%%%%%%%%%%%%%%%%%%%%%%%%%%%%%%%%%%%%%%%%%%%%%%%%%%
\problem{Ex 20.2 (BBT, ex 12:13.2).} Equip the space $C[0,1]$ with both the $L^1$ norm and the supremum norm. Show that the $L^1$ norm is bounded by a constant times the supremum norm. Show that the reverse is not true. Explain why the two results do not contradict the corollary to the open mapping theorem. \bigbreak

	\begin{proof}
		For any function $f \in C[0,1]$, define the constant function $g = \sup\limits_{x \in [0,1]}\{\vert f \vert \}$. Then $\norm{f}_\infty = \norm{g}_\infty$, but since $\vert f \vert  \leq g$, by monotonicity we have that $\norm{f}_1 \leq \norm{g}_1$. Thus $\norm{\cdot}_1$ is bounded by $1\norm{\cdot}_\infty$. \bigbreak
		
		Suppose that there were a constant $M$ such that $\norm{\cdot}_\infty$ is bounded by $M\norm{\cdot}_1$. Then consider the function $f(x) = x^{M} \in C[0,1]$. Then $\norm{f}_\infty = 1$, but $\norm{f}_1 = \tfrac{1}{1+M}$ so $M \norm{f}_1 = \tfrac{1}{1+M} < 1 = \norm{f}_\infty$.
	\end{proof}
	
	This does not violate the corollary to the Open Mapping Theorem because $C[0,1]$ under the $L^1$ norm is not a Banach Space. Consider the sequence of functions $f_n(x) = x^n$. This sequence converges pointwise to 
	$$
	f(x) =
	\begin{cases}
		0 \text{\ \ if $x<1$}\\
		1 \text{\ \ if $x=1$}
	\end{cases}
	$$
	Further, this sequence is Cauchy. Let $\epsilon > 0$. Pick $N$ to be an integer greater than $1/\epsilon$. Then for any $n,m>N$
	\begin{align*}
		\norm{x^n - x^m}_1 & = \int_0^1\vert x^n - x^m \vert dx \\
		& = \int_0^1 x^n - x^m dx \text{\ \ \ (WLOG $m>n$)}\\
		& = \tfrac{1}{n+1} - \tfrac{1}{m+1} \\
		& < \tfrac{1}{n+1} < \tfrac{1}{N} < \epsilon
	\end{align*}
	Since this sequence is Cauchy, but the pointwise limit is not in $C[0,1]$ this is not a Banach Space and the Open Mapping Theorm is not contradicted. 
\end{document}
