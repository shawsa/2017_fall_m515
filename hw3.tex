\documentclass[12pt]{article}


% Math		****************************************************************************************
\usepackage{fancyhdr} 
\usepackage{amsfonts}
\usepackage{amsmath}
\usepackage{amsthm}
\usepackage{dsfont}

% Commands	****************************************************************************************
%\newcommand{\problem}[1]{\hspace{-\widthof{#1}} \textbf{#1}}
\newcommand{\problem}[1]{\hspace{-4 ex} \large \textbf{#1}\\}
\newcommand{\reverseconcat}[3]{#3#2#1}

%page		****************************************************************************************
\usepackage[margin=1in]{geometry}
\usepackage{setspace}
\doublespacing
\pagestyle{fancy}
\fancyhf{}
\rhead{Shaw \space \thepage}
\setlength\parindent{0pt}


\begin{document}
	\thispagestyle{empty}
	
	\begin{flushright}
		Sage Shaw \\
		m515 - Fall 2017 \\
		\today
	\end{flushright}
	

\problem{5.1} Tao ex 1.2.1. Show that the countable union of Jordan measurable sets need not be Jordan measurable, even when bounded. Show that the countable intersection of Jordan measurable sets need not be Jordan measurable.

	\textit{Example:} Let $A = \{(x,y) \in [0,1] \times [0,1] \vert x \notin \mathbb{Q}\}$. Then $$A = ([0,1] \times [0,1]) \cap \bigcap\limits_{x\ \in \mathbb{Q}}([x,x]\times[0,1])$$ is a countable intersection of Jordan measurable sets (boxes in fact) that is not Jordan measurable since the upper Jordan measure is 1 and the lower Jordan measure is 0.




\problem{5.2} Tao ex 1.2.2. Give an example of a sequence of uniformly bounded, Riemann integrable functions on $[0,1]$ which converges pointwise to a function that is not Riemann integrable. Is it possible to give an example which converges uniformly?

	\textit{Example:} Let define $f_n:[0,1] \to \mathbb{R}$ for $n \in \mathbb{N}$ as follows: 
	\[
		f_n = 
			\begin{cases}
				0 \text{ if } nx \in \mathbb{Z} \\
				1 \text{ if } nx = \notin \mathbb{Z}
			\end{cases}
	\]
	Put simply, $f_n$ is a constant function with $n+1$ discontinuities at integer multiples of $1/n$. Clearly $\int f_n = 1$. See that $\{f_n\} \to f$ where
	\[
		f = 
			\begin{cases}
				0 \text{ if } x \in \mathbb{Q} \\
				1 \text{ if } x \notin \mathbb{Q}
			\end{cases}
	\]
	We demonstrated in class that $f$ is not Riemann integrable since $\underline{\int} f = 0$ and $\overline {\int} f = 1$.
	
	\textbf{Prove second part}



\problem{5.3} Show that $m^*(A)\leq m^{*j}(A)$. Give an example of a set $A$ such that $m^*(A)<m^{*j}(A)$.
	
	\begin{proof}
		Define $S_J(A)$ and $S_L(A)$ as follows:
		\begin{align*}
			S_J(A) & = \Bigg\{\sum_1^k\text{vol}B_i \Bigg \vert B_i \text{ are boxes and } A \subset \cup_1^k B_i \Bigg\} \\
			S_L(A) & = \Bigg\{\sum_1^\infty\text{vol}B_i \Bigg \vert B_i \text{ are boxes and } A \subset \cup_1^\infty B_i \Bigg\}
		\end{align*}
	Notice then that $m^{*J}(A) = \text{inf}(S_J(A))$ and $m^{*}(A) = \text{inf}(S_L(A))$. By padding the finite set of boxes with the empty set (which is a box) we can see that $S_J(A) \subseteq S_L(A)$ and thus $\text{inf}(S_L(A)) \leq \text{inf}(S_J(A))$. This should be clear since any lower bound of $S_L$ is also a lower bound of $S_J$. Substituting we see that $m^*(A)\leq m^{*j}(A)$. \\
	\end{proof}

	For an example where $m^*(A)<m^{*j}(A)$, consider $f: [0,1] \to \mathbb{R}$ where
	\[
		f = 
			\begin{cases}
				1 \text{ if } x \in \mathbb{Q} \\
				0 \text{ if } x \notin \mathbb{Q}
			\end{cases}
	\]
	Define the set $A = \{(x,y) \in [0,1] \times [0,1] \vert y \leq f(x)\}$ (i.e. the area under the curve). Since this can be contained by a union of a countable number of boxes with zero width, it's clear that $m^*(A) = 0$ however $m^{*J}(A) = 1$ as we discussed in class.


\problem{5.4} Show that Jordan outer measure does not satisfy countable subadditivity.

	\textit{Example:} consider the sets $A_n = [n,n+1] \times [0, 2^{-n}]$ for $n \in \mathbb{N}$. Clearly $m(A_n) = 2^{-n}$. Let $A = \bigcup\limits_{n \in \mathbb{N}}A_n$. Then either $m(A) = \infty$ or $m(A)$ does not exist, but $\sum\limits_{n \in \mathbb{N}}m(A_n) = 1$. That is that the countable sum of their measures is less than the measure of their countable union.



\problem{6.1}Tao ex 1.2.5. Suppose $A$ is expressible as a countable union of pairwise almost disjoint boxes. Show that $m^*(A)=m_{*j}(A)$



\end{document}
