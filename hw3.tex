\documentclass[12pt]{article}


% Math		****************************************************************************************
\usepackage{fancyhdr} 
\usepackage{amsfonts}
\usepackage{amsmath}
\usepackage{amsthm}
%\usepackage{dsfont}

% Commands	****************************************************************************************
%\newcommand{\problem}[1]{\hspace{-\widthof{#1}} \textbf{#1}}
\newcommand{\problem}[1]{\hspace{-4 ex} \large \textbf{#1}\\}
\newcommand{\reverseconcat}[3]{#3#2#1}

%page		****************************************************************************************
\usepackage[margin=1in]{geometry}
\usepackage{setspace}
\doublespacing
\pagestyle{fancy}
\fancyhf{}
\rhead{Shaw \space \thepage}
\setlength\parindent{0pt}


\begin{document}
	\thispagestyle{empty}
	
	\begin{flushright}
		Sage Shaw \\
		m515 - Fall 2017 \\
		\today
	\end{flushright}
	

\problem{5.1} Tao ex 1.2.1. Show that the countable union of Jordan measurable sets need not be Jordan measurable, even when bounded. Show that the countable intersection of Jordan measurable sets need not be Jordan measurable.

	\textit{Example:} Let $A = [0,1] \cap \mathbb{Q}$. Then $$A = \bigcup\limits_{x\ \in \mathbb{Q} \cap [0,1]}[x,x]$$ is a countable intersection of Jordan measurable sets (boxes in fact) that is not Jordan measurable since the upper Jordan measure is 1 and the lower Jordan measure is 0. \bigbreak

	\textit{Example:} Let $A_x = [0,1]\setminus\{x\}$. Clearly $m_j(A_x)=1$ for any $x$. Now consider $\bigcap\limits_{x \in \mathbb{Q}}A_x = [0,1] \setminus \mathbb{Q}$. The finite intersection of any $A_n$ has Jordan measure 1, but the countable intersection has lower Jordan measure 0 and upper Jordan measure 1 and is therefore not Jordan measurable.\\
	
	
\problem{5.2} Tao ex 1.2.2. Give an example of a sequence of uniformly bounded, Riemann integrable functions on $[0,1]$ which converges pointwise to a function that is not Riemann integrable. Is it possible to give an example which converges uniformly?

	\textit{Example:} Let define $f_n:[0,1] \to \mathbb{R}$ for $n \in \mathbb{N}$ as follows: 
	\[
		f_n = 
			\begin{cases}
				0 \text{ if } mx \in \mathbb{Z} \text{ for some positive integer }m \leq n\\
				1 \text{ else }
			\end{cases}
	\]
	Put simply, $f_n$ is a constant function with discontinuities at integer multiples of $1/m$ for each positive integer less than $n$. So $f_2$ has discontinuities at 0, $\frac{1}{2}$, and 1. The next function $f_3$ has those discontinuities (since $2<3$) as well as $\frac{1}{3}$ and  $\frac{2}{3}$. Clearly $\int f_n = 1$. See that $\{f_n\} \to f$ where
	\[
		f = 
			\begin{cases}
				0 \text{ if } x \in \mathbb{Q} \\
				1 \text{ if } x \notin \mathbb{Q}
			\end{cases}
	\]
	We demonstrated in class that $f$ is not Riemann integrable since $\underline{\int} f = 0$ and $\overline {\int} f = 1$. \bigbreak
	
	\textbf{Theorem}\\
	Let $\{f_n\}$ be a sequence of functions where each $f_n:[a,b] \to \mathbb{R}$ is Riemann integrable. If $\{f_n\}$ converges uniformly to $f:[a,b] \to \mathbb{R}$, then $f$ is Riemann integrable.
	\begin{proof}
		Let $\epsilon > 0$ be given. Then there exists some $N \in \mathbb{N}$ such that if $n \in \mathbb{N}$ and $N \leq n$, then $\vert f_n(x) - f(x) \vert < \text{min}\Big\{\frac{\epsilon}{3}, \frac{\epsilon}{3(b-a)}\Big\}$ by definition of uniform convergence. Since $f_n$ is Riemann integrable, we know there exists piecewise constant functions $\underline{h}_n$ and $\overline{h}_n$ such that $\underline{h}_n \leq f_n \leq \overline{h}_n$ and $\textit{pc}\int \overline{h}_n - \underline{h}_n < \frac{\epsilon}{3}$. Then
		$$
		\underline{h}_n - \frac{\epsilon}{3(b-a)} \leq f_n - \frac{\epsilon}{3(b-a)} \leq f \leq f_n + \frac{\epsilon}{3(b-a)} \leq \overline{h}_n +\frac{\epsilon}{3(b-a)}
		$$
		and 
		\begin{align*}
			\textit{pc}\int (\overline{h}_n +\frac{\epsilon}{3(b-a)}) & - (\underline{h}_n - \frac{\epsilon}{3(b-a)}) \\
			& = \frac{2}{3}\epsilon + \textit{pc}\int \overline{h}_n - \underline{h}_n \\
			& < \frac{2}{3}\epsilon + \frac{1}{3}\epsilon = \epsilon
		\end{align*}
		Thus $f$ is Riemann integrable.\\
	\end{proof}



\problem{5.3} Show that $m^*(A)\leq m^{*j}(A)$. Give an example of a set $A$ such that $m^*(A)<m^{*j}(A)$.
	
	\begin{proof}
		Define $S_J(A)$ and $S_L(A)$ as follows:
		\begin{align*}
			S_J(A) & = \Bigg\{\sum_1^k\text{vol}B_i \Bigg \vert B_i \text{ are boxes and } A \subset \cup_1^k B_i \Bigg\} \\
			S_L(A) & = \Bigg\{\sum_1^\infty\text{vol}B_i \Bigg \vert B_i \text{ are boxes and } A \subset \cup_1^\infty B_i \Bigg\}
		\end{align*}
	Notice then that $m^{*J}(A) = \text{inf}(S_J(A))$ and $m^{*}(A) = \text{inf}(S_L(A))$. By padding the finite set of boxes with the empty set (which is a box) we can see that $S_J(A) \subseteq S_L(A)$ and thus $\text{inf}(S_L(A)) \leq \text{inf}(S_J(A))$. This should be clear since any lower bound of $S_L$ is also a lower bound of $S_J$. Substituting we see that $m^*(A)\leq m^{*j}(A)$. \\
	\end{proof}
	
	For an example where $m^*(A)<m^{*j}(A)$, consider $A = [0,1] \cap \mathbb{Q}$. It can be shown that $m^{*j}(A)=1$. Also $A \subseteq \bigcup\limits_{x \in \mathbb{Q}}[x,x]$, but $m_j([x,x])=0$. Then $m^*(A) = \sum\limits_{x \in \mathbb{Q}}m_j([x,x]) = 0 < 1 = m_j(A)$. 


\problem{5.4} Show that Jordan outer measure does not satisfy countable subadditivity.

	\textit{Example:} consider the sets $A_n = [n,n+1] \times [0, 2^{-n}]$ for $n \in \mathbb{N}$. Clearly $m(A_n) = 2^{-n}$. Let $A = \bigcup\limits_{n \in \mathbb{N}}A_n$. Then $\sum\limits_{n \in \mathbb{N}}m(A_n) = 1$, but consider $m^{*J}(A)$. Since $A$ is unbounded, any box that covers it will be unbounded as well. Since every $A_n$ has a height this unbounded box must also have a height in order to cover $A$. Clearly every elementary set that covers $A$ has infinite measure and $m(A) = \infty$ or $m(A)$ does not exist and the countable sum of their measures is less than the measure of their countable union.



\problem{6.1}Tao ex 1.2.5. Suppose $A$ is expressible as a countable union of pairwise almost disjoint boxes. Show that $m^*(A)=m_{*j}(A)$ 
	\begin{proof}
		Clearly $m_{*j}(A) \leq m^*(A)$ since every covering of $A$ contains $A$, and $A$ contains every set that it covers. \bigbreak
		
		From our preminse assume that $A = \bigcup\limits_{n \in \mathbb{N}}^\infty B_n$ where $B_n$ is a box and all the boxes are pairwise disjoint. By definition $m^*(A) \leq \sum\limits_{n=1}^\infty m(B_n)$. \bigbreak
		
		Then define $S_n = \bigcup\limits_{n=1}^k B_n$ is an elementary set and $S_n \subseteq A$ for every $n$. Note that since these boxes are almost disjoint, the measures of their pairwise intersections is $0$, thus 
		$$m(S_n) = m\Big(\bigcup\limits_{n=1}^k B_n \Big) = \sum\limits_{n=1}^k m(B_n)$$
		by finite additivity of elementary measure.
		
		Consider the sequence $\{m(S_n)\}$. Note that Since elementary measure is non-negative, this sequence is monotonically increasing. By definition it it bounded above by  $m_{*j}(A)$. Since $\lim\limits_{n \to \infty} m(S_n) = \sum\limits_{n=1}^\infty m(B_n) = m^*(A)$, we can say that $m^*(A) \leq m_{*j}(A)$. \bigbreak
		
		Thus $m^*(A)=m_{*j}(A)$.		
	\end{proof}


\end{document}
